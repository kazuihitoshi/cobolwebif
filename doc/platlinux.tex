\subsubsection{WebサーバApacheの設定}

ほとんどのLinuxディストリビューションではApacheのインストールが完了しているので、
当資料ではインストールについては記述しません。

インストール後に必要な設定のみ記述します。

当資料のプログラム環境を整えるためにはApacheに、CGI実行,ユーザディレクトリ,別名定義モジュールを組み込む必要がある。

\vspace{1em}
{\gt 組み込みが必要なApacheモジュール}

\begin{tabular}{|l|l|}
\hline
モジュール名 & モジュールファイル名 \\
\hline
CGI実行モジュール		&mod\_cgi.so	\\
ユーザディレクトリモジュール	&mod\_userdir.so\\
別名定義モジュール		&mod\_alias.so	\\
\hline
\end{tabular}

実際の設定は以下の様になります。(以下の設定例ではapacheのバージョンを1.3としています。)

FileName:/etc/apache/httpd.conf

\begin{tabular}{|l|}
\hline
\verb+LoadModule cgi_module /usr/lib/apache/1.3/mod_cgi.so+\\
\verb+LoadModule userdir_module /usr/lib/apache/1.3/mod_userdir.so+\\
\verb+LoadModule alias_module /usr/lib/apache/1.3/mod_alias.so+\\
\verb+AddHandler cgi-script .cgi+\\
\verb+<Directory /home/cobol/public_html/cgi-bin>+\\
\verb+Options +ExecCGI+\\
\verb+</Directory>+\\
\hline
\end{tabular}

または

FileName:/usr/local/etc/apache/httpd.conf

\begin{tabular}{|l|}
\hline
\verb+LoadModule cgi_module /usr/local/lib/apache/1.3/mod_cgi.so+\\
\verb+LoadModule userdir_module /usr/local/lib/apache/1.3/mod_userdir.so+\\
\verb+LoadModule alias_module /usr/local/lib/apache/1.3/mod_alias.so+\\
\verb+AddHandler cgi-script .cgi+\\
\verb+<Directory /home/cobol/public_html/cgi-bin>+\\
\verb+Options +ExecCGI+\\
\verb+</Directory>+\\
\hline
\end{tabular}

{\gt ユーザ登録}

COBOL開発用のユーザを登録する必要はありませんが、
当資料の便宜上として{\gt cobol}ユーザを登録します。

以下のコマンドは、root(スーパユーザ)にて行ってください。

\begin{tabular}{|l|}
\hline
\verb&# useradd -m -d /home/cobol cobol&\\
\verb&# passwd cobol&\\
\verb&Enter new UNIX password: &\\
\verb&Retype new UNIX password: &\\
\verb&passwd: password updated successfully&\\
\hline
\end{tabular}
Enter new UNIX password:にcobolユーザのパスワードを入力してください。

Retype new UNIX password:にはEnter new UNIX passwordにて入力したパスワードをもう一度入力してください。

CGIファイルを格納するディレクトリを作成します。

\begin{tabular}{|l|}
\hline
\verb&mkdir -p /home/cobol/public_html/cgi-bin&\\
\verb&chown cobol:cobol /home/cobol/public_html/cgi-bin&\\
\verb&chown cobol:cobol /home/cobol/public_html&\\
\verb&chmod 755 /home/cobol/public_html/cgi-bin&\\
\verb&chmod 755 /home/cobol/public_html&\\
\hline
\end{tabular}

 
\subsubsection{COBOLコンパイラのインストール}

当資料で使用するCOBOLコンパイラOPEN COBOL,TinyCobolのインストール方法を紹介します。

\begin{itemize}

\item{OPEN COBOLのインストール}

OPEN COBOLは{\it http://www.open-cobol.org/}よりダウンロード出来ます。

{\it http://www.open-cobol.org/}より{\it open-cobol-0.10.tar.gz}をダウンロードし、以下の手順でコンパイルします。

\begin{tabular}{|l|}
\hline
\verb+# tar zxvf open-cobol-0.10.tar.gz+\\
\verb+# cd open-cobol-0.10+\\
\verb+# ./configure+\\
\verb+# make+\\
\verb+# su+\\
\verb+Password: スーパユーザのパスワードを入力+\\
\verb+# make install+\\
\hline
\end{tabular}

以下のファイルがインストールされます。

\begin{tabular}{ll}
/usr/local/bin/cobc           &COBOLコンパイラ\\
/usr/local/bin/cobpp          &COBOLプリプロセッサ\\
/usr/local/lib/libcob.a       &COBOLライブラリアーカイブ \\
/usr/local/lib/libcob.la      &COBOLライブラリlibtool用ファイル\\
/usr/local/lib/libcob.so      &COBOL共有ライブラリ\\
/usr/local/lib/libcob.so.0    &COBOL共有ライブラリ\\
/usr/local/lib/libcob.so.0.0.0&COBOL共有ライブラリ\\
\end{tabular}

注意:./configureにてエラーが発生した場合、必要なモジュールを組み込んでください。

\item{TinyCobolのインストール}

TinyCobolは{\it http://tiny-cobol.sourceforge.net/}よりダウンロード出来ます。

{\it http://tiny-cobol.sourceforege.net/}より{\it tinycobol-0.58.tar.gz}をダウンロードし、以下の手順でコンパイルします。

\begin{tabular}{|l|}
\hline
\verb+# tar zxvf tinycobol-0.58.tar.gz+\\
\verb+# cd tinycobl-0.58+\\
\verb+# ./configure+\\
\verb+# make+\\
\verb+# su+\\
\verb+Password: スーパユーザのパスワードを入力+\\
\verb+# make install+\\
\hline
\end{tabular}

以下のファイルがインストールされます。

\begin{tabular}{ll}
/usr/local/bin/htcobol            &COBOLコンパイラ\\
/usr/local/bin/cobpp              &COBOLプリプロセッサ\\
/usr/local/man/man1/htcobol.1     &オンラインマニュアル\\
/usr/local/man/man1/htcobf2f.1    &オンラインマニュアル\\
/usr/local/man/man1/htcobolpp.1   &オンラインマニュアル\\
/usr/local/lib/libhtcobol.a       &COBOLライブラリアーカイブ\\
/usr/local/lib/libhtcobol2.a      &COBOLライブラリアーカイブ\\
\end{tabular}

注意:./configureにてエラーが発生した場合、必要なモジュールを組み込んでください。

\end{itemize}

\subsubsection{COBOL Webインタフェースのインストール}

当資料で紹介しているCOBOL Webインタフェースをインストールします。
COBOL Webインタフェースはソースモジュールによる配布としております。

{\it http://pulse.homeip.net/webif.html}より{\it web.0.3.tar.gz}をダウンロードします。

\$ tar zxvf web.0.3.tar.gz

\$ make libweb.a

\$ make install

以上でインストールは終了です。






