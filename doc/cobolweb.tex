\documentstyle[a4j]{jarticle}
%\documentstyle{jarticle}
%\documentstyle{jbook}
%\documentstyle{jreport}
\begin{document}
\date{2003/3/23}

\title{COBOLによるCGIプログラムAPI}
\author{数井 仁}

\maketitle

\tableofcontents

\newpage

\begin{tabular}{|l|l|l|}
\hline
0.5&サンプルソースにチャットを追加&2003/3/23\\
\hline
0.4&各種修正中&2003/2/23〜2003/3/23\\
\hline
0.3&ロック関数の追加に伴い、ドキュメントにも追加 &2003/2/16\\
   &サブプログラム起動のサポートに伴い、ドキュメントにも追加 & \\
\hline
0.2&開発環境の構築方法を追加 & 2003/2/4\\
\hline
0.1&0.1版に対応するドキュメントとして作成&2003/1/7\\
\hline
\end{tabular}
\newpage

\section{概要}

COBOLは一昔前の言語であることは否めませんが、現在も事務処理を行うには適した特徴をたくさん持っております。

当資料及びAPIは、この古めかしさも否めないCOBOLをWebベースで使ってしまおうという狙いを持っています。

本来であれば、COBOLを知らない方にとっても有用な資料としたいところですが、今回は多少なりともCOBOLを知っている方に絞って記述します。

COBOLプログラムをWebから使うのは実は非常に簡単です。

その理由は、WebのCGI設計者が既存のプログラムを出来るだけ変更無しに使えるように設計されているためです。

ちょっと具体的な点に触れてみましょう。

WebのCGIはWebブラウザとWebサーバの2人三脚で動作します。

Webブラウザからの送信情報はWebサーバへ送られ、該当するプログラムを起動します。

\par

プログラムは処理をHTMLで標準出力へ出力します。

標準出力へ出力されたHTMLは、Webサーバを通してWebブラウザへ返信され、表示がなされます。

\vspace{1em}

\begin{tabular}{|lllll|}
\hline
クライアント側 & & サーバ側 & & \\
\hline
Webブラウザ & → & Webサーバ & → & プログラム \\
            &    &           &    &  Webブラウザから送信された内容を受信 \\
            &    &           &    &  標準入力と環境変数のどちらかで受信する\\
            &    &           &    &       ↓ \\
            &    &           &    &  プログラム内部で処理\\
            &    &           &    &       ↓ \\
            &    &           &    &  処理結果をHTMLで表示(標準出力へ出力)\\
返信内容が表示される& ← &  ←       &←  &  ←表示と同時に表示内容が返信される \\
\hline
\end{tabular}

\vspace{1em}

どうでしょう。COBOLでどのようにコーディングすればよいかイメージ出来た方もいらっしゃるのではないでしょうか?

\par

そう、基本的にはacceptして処理してdisplayすればよいのです。

おお、そうなんだと思われた方はちょっと待ってください。
Webブラウザからの送信情報のうち日本語等はエンコードされていたり、
COBOLでそのまま使うには少々困難な情報です。
といっても、がっかりしないでください、この資料で紹介するWebAPIを使えば非常に簡単に扱いやすい形で行うことが出来ます。

当資料で使用するCOBOL処理系はOpenCOBOL\footnote{OpenCOBOLは西田氏作成のフリーコンパイラです。}または TinyCobol\footnote{TinyCobolはAndrewCameron氏他作成のフリーコンパイラです}です。
OSはUNIXクローン(筆者の環境はDebianLinux Woody)やMigGwを利用したWindowsでCOBOL処理系とApacheが動作する環境であれば
どんな環境でも動作するとおもいます。

この資料とWeb APIにて、一人でも多くの方にCOBOLでWebプログラミングをお楽しみ頂けることを望んでおります。

それでは本編をお楽しみください。









\section{CGIにおけるCOBOLプログラムの振る舞い}
COBOLプログラムは通常{\gt(1)オンライン処理,(2)バッチ処理}の2種類があります。
(1)オンライン処理には2種類の方式がありますので、以下の3種類が存在しております。

{\gt (1)オンライン処理 端末占有タイプ}のCOBOLプログラムは起動されると端末を占有して動作し、COBOLプログラム自身が対話形式で動作します。
このプログラムは指定された処理が完全に完了するか、終了の命令が与えられるまで動作し続けます。

{\gt (1')オンライン処理 OLTPタイプ}のCOBOLプログラムは起動時に動作パラメータ(以後メッセージ)を受け付け、後はそのパラメータに従って動作をおこない、結果を戻します。
動作はこのサイクルで完結します。

{\gt (2)バッチ処理}のCOBOLプログラムは起動時に動作パラメータを受け付け、後はそのパラメータに従って動作を行います。端末を占有して動作状況を逐一表示するタイプのものも有りますが、構造としては起動時のパラメータに従って最後まで動作をするだけのものです。

では、COBOL Web APIではどのようになるでしょう?

COBOL Web APIでは上記の{\gt(1')オンライン処理OLTPタイプ,(2)バッチ処理}\footnote{現時点ではバッチ処理の機能はありません。別途バッチスプーラなる機能を追加し、そちらでサポートすることを考えております。}の方式で動作します。


長々と記述しましたが、Webブラウザから送信されるメッセージは、結果の返信によって完結する振る舞いをしなければなりません。

\vspace{1em}
\begin{tabular}{|lllll|}
\hline
クライアント側 & & サーバ側 & & \\
\hline
Webブラウザ & → & Webサーバ & → & プログラム \\
            &    &           &    &  Webブラウザから送信された内容を受信 \\
            &    &           &    &  標準入力と環境変数のどちらかで受信する\\
            &    &           &    &       ↓ \\
            &    &           &    &  COBOLプログラム実行\\
            &    &           &    &       ↓ \\
            &    &           &    &  処理結果をHTMLで表示、STOP RUN\\
返信内容が表示される& ← &  ←       &←  &  ←表示と同時に表示内容が返信される \\
\hline
\end{tabular}

%もし{\gt (1)オンライン処理 端末占有タイプ}しかイメージがわかないという方(特にパソコンやオフコン上のCOBOLはこちらの概念を使用するものが多いので)はサンプルから見てください。
%きっと理解していただけます。

%次よりWebインタフェースAPIの説明を記述します。

\vspace{1em}

では、ちょっと具体例に触れてみましょう。WebAPIを使うと以下の様なプログラムになります。

\begin{verbatim}
                :          :
      WORKING-STORAGE SECTION.
      01 WEBIF-REC.
        02 FILLER PIC X(20) VALUE 'DATA1'.
                :          :
        02 FILLER PIC X VALUE LOW-VALUE.
        02 FILLER PIC X VALUE LOW-VALUE.
      PROCEDURE DIVISION.
      MAIN-PROC SECTION.
      001.
          CALL 'WEB_GET_QUERY_STRING'.         ←ここでWebブラウザからの情報を取得
          CALL 'WEB_POP' USING WEBIF-REC.      ←COBOLの変数領域にWebブラウザからの
                                                 情報をWEBIF-RECに展開

          処理を行う。                         ←処理を行います

          CALL 'WEB_PUSH' USING WEBIF-REC.      ←Webブラウザ返信用にCOBOL変数領域の
                                                  内容を記憶させる
          CALL 'WEB_SHOW'.                     ←WEB_PUSHで記憶させた変数とHTML
                                                 ファイルを合成して
                                                 Webブラウザへ返信
      999.
            STOP RUN.
\end{verbatim}
      

Webブラウザから送信された内容を元に処理をおこなってWebブラウザへ返信します。

セッションの情報はそこで忘れ去られてしまいますが、次回必要な変数などはWEB\_PUSHで覚えさせておけば
Webブラウザに記憶させるようになっています。

このWebブラウザへ記憶させる方法はhidden属性を付与したtextで記憶させるようになっています。
\footnote{今後クッキーに対応予定です。}

ちょっとは感覚が解ってもらえたのではないでしょうか?

それでは次ページよりWebAPIの解説を記述します。

もっと具体例から理解したいという方は、WebAPIの解説を読み飛ばしてサンプルの章を参照してください。

%\begin{varbatim}
%
%\end{varrbatim}




\section{WebインタフェースAPI}
CGIにてWebブラウザから送信されるデータは以下の様な形式です。

送信されたデータが

\begin{tabular}{|l|l|}
\hline
変数名 & 内容\\
\hline
d1     &あいうえお \\
d2     &かきくけこ\\
\hline
\end{tabular}

の内容の場合、送信データは

\begin{tabular}{|l|}
\hline
\verb+d1=%A4%A2%A4%A4%A4%A6%A4%A8%A4%AA&d2=%A4%AB%A4%AD%A4%AF%A4%B1%A4%B3+\\
\hline
\end{tabular}

となります。(ページの日本語コードをeucコードと仮定しています)

送信データをCOBOLプログラム的に上記を扱いやすい形式に編集するには「\&」で分割し、

\begin{tabular}{|l|}
\hline
\verb+d1=%A4%A2%A4%A4%A4%A6%A4%A8%A4%AA+\\
\hline
\verb+d2=%A4%AB%A4%AD%A4%AF%A4%B1%A4%B3+\\
\hline
\end{tabular}

\vspace{1em}

\hspace{-1em}\%付でデコードされた情報をエンコードする必要があります。

\begin{tabular}{|l|}
\hline
\verb+d1=%A4%A2%A4%A4%A4%A6%A4%A8%A4%AA→a4a2a4a4a4a6a4a8a4aa(あいうえお)+\\
\hline
\verb+d2=%A4%AB%A4%AD%A4%AF%A4%B1%A4%B3→a4aba4ada4afa4b1a4b3(かきくけこ)+\\
\hline
\end{tabular}

\vspace{1em}

COBOLで試しにコーディングしてみると

\begin{verbatim}
   :       :
 working-storage section.
 01 query-string-rec.
   02 d1   pic x( .. ).
   02 d2   pic x( .. ).
 01 query-string-data.   ←このエリアに送信データが入ると仮定
   02 filler pic x( .. ). 
 procedure division.
 main  section.
 00.
   unstring query-string-data delimited by "&"
    into d1 d2
   end-unstring   
   d1をエンコード
   d2をエンコード
   :       :
   本来の処理を実行
   :       :
   display 'Content-Type:text/html'      ←結果をHTMLで表示
   display ''
   display '<html>'
      :      :
   display '</html>'
 99.
   :       :
\end{verbatim}

かなり省略してしまいましたが(エンコードの所など)、だいたい上記の様になります。

この方法は全部COBOLソースで記述出来てしまうので、COBOLさえ動けば何とかすることが出来るでしょう。

ですが、ここでもうちょっとひねってみてください。上記だと入力項目毎にエンコード行を書いていかないといけません。
1〜10項目ぐらいであれば上記も良いでしょうが、入力項目が多くなると少々つらくなってきます。
COBOLプログラムの中にHTMLが入り込んでいるのも(別言語が混ざっている感じがして)、少々きになります。

筆者としては、COBOLは階層化されたデータ構造をまとめてえいやーーと処理出来てしまうあのダイナミック(がさつさ?)さを捨てたくありません。
また、HTMLは表示様式であるためプログラムとは別にしておいた方がモジュールの独立性上、良いようにおもわれます。

たとえば、処理をプログラマが作成してHTMLはWebデザイナー専門の人に作ってもらう等、それぞれが独立していれば分業が可能です。

一人の作業時も処理実装時には簡易的なHTMLで実装し、後で見た目(HTML)だけを作成し直すことも可能なのです。
(逆もまたしかりです)

ということで筆者はCOBOLをWebで使用するためのAPIを作成しました。

このAPIを使用すれば、

\begin{itemize}
\item{ エンコード,デコード }
\item{HTMLファイルとCOBOL出力データの合成}
\item{内部変数のページへの保存(hidden表示)}
\end{itemize}

等、実際の処理とは関係ない部分を代わりに処理してもらえます。


\subsection{WEB\_GET\_QUERY\_STRING}

\begin{itemize}
\item{書式}

CALL 'WEB\_GET\_QUERY\_STRING'

\item{機能}

CGIで渡されるデータをWeb APIグローバル領域に読み込みます。

CGIのデータ転送方法{\it GET,POST}の両方に対応しています。

{\footnotesize
\begin{tabular}{lcl}
GET & ・・・ & 環境変数QUERY\_STRINGにより渡される方式 \\
POST & ・・・ &標準入力で渡される方式\\
\end{tabular}
}

\vspace{1em}

Web APIグローバル領域の変数はWEB\_POP関数にてCOBOL変数領域へ取り出す事が出来ます。


\end{itemize}
\subsection{WEB\_PUT\_QUERY\_STRING}

\begin{itemize}
\item{書式}

CALL 'WEB\_PUT\_QUERY\_STRING'

\item{機能}

Web APIグローバル領域の内容を環境変数QUERY\_STRINGに設定します。

別プログラムを起動する場合に引数の引き渡しとして使う事が可能である。

この関数は廃止しました。同様の機能がWEB\_PUSHで実装済みなので、そちらを使用ください。

\end{itemize}

\subsection{WEB\_POP,WEB\_PUSH変数領域}

Web APIグローバル領域とCOBOL変数領域の引き渡し空間を定義します。

変数領域は{\it WORKING-STRAGE SECTION.}に定義します。

定義方式は

変数名称1 + NULL + 変数属性1 + NULL + 変数領域1 + NULL + 変数名称2 + NULL + 変数属性2 + NULL + 変数領域2 + NULL .... 変数名称n + NULL + 変数属性n + NULL + 変数領域n + NULL + NULL

のメモリイメージになるように変数領域を記述する。最後のNULLは2つ連続していなければなりません。


変数属性は今後の拡張領域のため、現時点ではスペースを記述しておけば問題ありません。

将来的には左寄せ(デフォルトは右寄せ)や、数値のみ入力などの制御が可能となる予定です。

コーディング書式は次のようになります。

{\gt コーディング書式}

{\footnotesize
\begin{tabular}{|l|}
\hline
 working-strage section.                                              \\
 01 web-if-rec.                                                       \\
   02  filler pic x(変数名称のデータサイズ) value '変数名称'.           \\
   02  filler pic x     value low-value.                              \\
   02  filler pic x     value space.                                  \\
   02  filler pic x     value low-value.                              \\
   02 変数名のデータ領域名 pic x(変数名データサイズ) value space.       \\
   02  filler pic x     value low-value.                              \\
   02  filler pic x(変数名称2データサイズ) value '変数名称2'.       \\
   02  filler pic x     value low-value.                              \\
   02  filler pic x     value space.                                  \\
   02  filler pic x     value low-value.                              \\
   02 変数名2のデータ領域名  pic x(変数名2データサイズ) value space.      \\
   02  filler pic x     value low-value.                              \\
            :       :                                                 \\
   02  filler pic x(変数名nの名称データサイズ) value '変数名n'.       \\
   02  filler pic x     value low-value.                              \\
   02  filler pic x     value space.                                  \\
   02  filler pic x     value low-value.                              \\
   02 変数名nのデータ領域名  pic x(変数名nデータサイズ) value space.      \\
   02  filler pic x     value low-value.                              \\
   02 filler pic x     value low-value.                               \\
\hline
\end{tabular}
}

\vspace{0.5em}

{\small
追記事項:

\begin{itemize}
\item{変数名称 ・・・半角英数字のみで表記、変数名の英字は大文字小文字の区別は無い。}
\item{変数名のデータ領域名 ・・・実データが保存される領域。}
\end{itemize}
}

{\gt 例 CGIよりNAME,EMAIL変数が渡される場合のコーディング例}

\vspace{0.5em}

{\footnotesize
\begin{tabular}{|l|r|l|}
\hline
入力項目名 & 桁数 & 変数名 \\\hline
ATAI1       &  12  & ATAI1  \\\hline
ATAI2       &  12  & ATAI2  \\\hline
KOTAE       &  13  & KOTAE  \\
\hline
\end{tabular}
}
\vspace{1em}

コーディング例は以下の様になります。

\vspace{0.5em}

{\footnotesize
\begin{tabular}{|l|}
\hline
\verb+working-strage section.+\\
\verb+01 web-if-rec.+\\
\verb+ 02 filler pic x(5)  value 'ATAI1'.+\\
\verb+ 02 filler pic x     value low-value.+\\
\verb+ 02 filler pic x     value space.+\\
\verb+ 02 filler pic x     value low-value.+\\
\verb+ 02 ATAI1  pic x(12) value space.+\\
\verb+ 02 filler pic x     value low-value.+\\
\verb+ 02 filler pic x(5)  value 'ATAI2'.+\\
\verb+ 02 filler pic x     value low-value.+\\
\verb+ 02 filler pic x     value space.+\\
\verb+ 02 filler pic x     value low-value.+\\
\verb+ 02 ATAI2  pic x(12) value space.+\\
\verb+ 02 filler pic x     value low-value.+\\
\verb+ 02 filler pic x(5)  value 'KOTAE'.+\\
\verb+ 02 filler pic x     value low-value.+\\
\verb+ 02 filler pic x     value space.+\\
\verb+ 02 filler pic x     value low-value.+\\
\verb+ 02 KOTAE  pic x(13) value space.+\\
\verb+ 02 filler pic x     value low-value.+\\
\verb+ 02 filler pic x     value low-value.+\\
\hline
\end{tabular}
}

このWEB\_POP,WEB\_PUSH変数領域は、実際には以下の様にして使用します。
\vspace{1em}

{\footnotesize
\begin{tabular}{|l|}
\hline
\verb+working-strage section.+\\
\verb+01 web-if-rec.+\\
\verb+ 02 filler pic x(5)  value 'ATAI1'.+\\
\verb+ 02 filler pic x     value low-value.+\\
\verb+ 02 filler pic x     value space.+\\
\verb+ 02 filler pic x     value low-value.+\\
\verb+ 02 ATAI1  pic x(12) value space.+\\
\verb+ 02 filler pic x     value low-value.+\\
\verb+ 02 filler pic x(5)  value 'ATAI2'.+\\
\verb+ 02 filler pic x     value low-value.+\\
\verb+ 02 filler pic x     value space.+\\
\verb+ 02 filler pic x     value low-value.+\\
\verb+ 02 ATAI2  pic x(12) value space.+\\
\verb+ 02 filler pic x     value low-value.+\\
\verb+ 02 filler pic x(5)  value 'KOTAE'.+\\
\verb+ 02 filler pic x     value low-value.+\\
\verb+ 02 filler pic x     value space.+\\
\verb+ 02 filler pic x     value low-value.+\\
\verb+ 02 KOTAE  pic x(13) value space.+\\
\verb+ 02 filler pic x     value low-value.+\\
\verb+ 02 filler pic x     value low-value.+\\
\verb+01 web-in-rec.+\\
\verb+  02 ATAI1   pic 9(12).+\\
\verb+  02 ATAI2   pic 9(12).+\\
\verb+  02 KOTAE   pic 9(12).+\\
\verb+01 web-ot-rec.+\\
\verb+  02 ATAI1   pic ZZZZZZZZZZZ9.+\\
\verb+  02 ATAI2   pic ZZZZZZZZZZZ9.+\\
\verb+  02 KOTAE   pic ZZZZZZZZZZZ9.+\\
\verb+ procedure division.+\\
\verb+    call 'WEB_GET_QUERY_STRING'+\\
\verb+    call 'WEB_POP' using web-if-rec.+\\
\verb+    move corr web-if-rec to web-in-rec.+\\
  処理 サンプルなので足し算を行う\\
\verb$    compute KOTAE of web-in-rec = ATAI1 of web-in-rec + ATAI2 of web-in-rec.$\\
  出力用に数値を編集\\
\verb+   move corr web-in-rec to web-ot-rec.+\\
WEBインタフェース領域へコピー\\
\verb+   move corr web-ot-rec to web-if-rec.+\\
\verb+   call 'WEB_PUSH' using web-if-rec.+\\
\verb+   call 'WEB_SHOW'.+\\
\verb+   stop run.+\\
\hline
\end{tabular}
}

流れを示すと、

{\small

\begin{itemize}

\item[(1)]{CGI変数を受け取る。}
\item[(2)]{web-if-rec エリアにCGIから受け取った値を格納する。}
\item[(3)]{web-if-rec エリアを処理で扱いやすい web-in-rec へコピーする。}
\item[(4)]{web-in-recをベースに処理を行う。}
\item[(5)]{web-in-recの内容を出力形式編集領域web-ot-recへコピーする。}
\item[(6)]{web-ot-recをCGIエリアweb-in-recへ転送する。(web-in-recには所定の様式編集がなされた値が入る)}
\item[(7)]{web-in-recを元にHTMLを合成する。}
\end{itemize}
}

となります。

\subsection{WEB\_POP}

\begin{itemize}
\item{書式}

CALL 'WEB\_POP' using WEB\_POP,WEB\_PUSH変数領域

\item{機能}

Web APIグローバル領域のデータをWEB\_POP,WEB\_PUSH変数領域に取り出します。

\vspace{1em}

WEB\_POP,WEB\_PUSH変数領域に変数名が定義されていないデータはWeb APIグローバル領域から取り出されません。

\end{itemize}
\subsection{WEB\_PUSH}

\begin{itemize}
\item{書式}

CALL 'WEB\_PUSH' using WEB\_POP,WEB\_PUSH変数領域

\item{機能}

WEB\_POP,WEB\_PUSH変数領域の内容をWeb APIグローバル領域に設定します。

\vspace{1em}

\end{itemize}
\subsection{WEB\_SHOW}

\begin{itemize}
\item{書式}

CALL 'WEB\_SHOW'

\item{機能}

Web APIグローバル領域のデータとHTMLを合成して出力します。

\vspace{1em}

外部HTMLファイルはWeb APIグローバル領域のSCREENNAME変数の内容を使用します。

\end{itemize}
\subsection{WEB\_LOCK,WEB\_UNLOCK}

\begin{itemize}
\item{書式}

{\gt ロック機能コピー句の読込}

\begin{tabular}{|l|}
\hline
WORKING-STORAGE SECTION.\\
COPY web-lock.\\
77  rc  pic s9(10).\\
\hline
\end{tabular}

{\gt 排他ロック}

\begin{tabular}{|l|}
\hline
MOVE 'ロックID' TO LC-LOCK-FILE.\\
SET LC-LOCK-EXCLUSIVE   TO TRUE.\\
CALL 'WEB\_LOCK' using LC-WEB-LOCK rc.\\
\multicolumn{1}{|c|}{: :}\\
CALL 'WEB\_UNLOCK' using LC-WEB-LOCK rc.\\
\hline
\end{tabular}

{\gt 共有ロック}

\begin{tabular}{|l|}
\hline
MOVE 'ロックID' TO LC-LOCK-FILE.\\
SET LC-LOCK-SHARE       TO TRUE.\\
CALL 'WEB\_LOCK' using LC-WEB-LOCK rc.\\
\multicolumn{1}{|c|}{: :}\\
CALL 'WEB\_UNLOCK' using LC-WEB-LOCK rc.\\
\hline
\end{tabular}

\item{機能}

ロック機能を実現する。ファイル更新の同期にはロック機能は不可欠です。

当機能により、複数プロセスによる同時ファイル更新によるファイルクラッシュを防ぐ事が出来ます。

通常ロックというとファイルなどの個別リソースのロックを想像しがちですが、
CGIプログラム自体をトランザクションと考え、トランザクション単位のロックの考え方をします。

同期の必要なCGIプログラムは基本的に同一のディレクトリで動作する事を想定しており、
上記関数のロックIDはロック制御用ファイルのファイル名として取り扱います。

{\footnotesize
\begin{tabular}{|l|}
\hline
MOVE 'TR' TO LC-LOCK-FILE.      \\
SET LC-LOCK-EXCLUSIVE TO TRUE.  \\
CALL 'WEB\_LOCK' using LC-WEB-LOCK rc.\\
\hline
\end{tabular}
}

とのコードを実行すると。CGIプログラムのカレントディレクトリに''TR''というファイルを作成し、
flockにて排他ロックがかかります。

これで、''TR''というロックIDで排他ロックをかけるCGIプログラムはアンロックされるまで実行を待たされます。

(尚、ロックしたプログラムがアンロック前にクラッシュしてもロックリソースが解放されるので、自動的にアンロック状態となり、プログラムのバグなどによるシステムデッドロックは発生しません。)

排他ロックとは書き込み禁止の時に使用するロックであり、共有ロックとは読込中に書き込みを禁止するロックです。

トランザクションによってロックIDをなににするか、ロックモードをどうするかは、プログラム設計者がきっちり決める必要があります。

また、ロック制御用のファイルを作成することから、初回ロックにはファイルの作成権が必要(初回はロックファイルを生成し、アンロック後もロック制御ファイルは保持される)となります。
(排他ロックは2回目のロック時も別にロック制御ファイルに対する書き込み権が必要です。)

\vspace{1em}


\end{itemize}




\subsection{WEB\_SET\_COOKIE,WEB\_POP\_COOKIE}

\begin{itemize}
\item{書式}

{\gt COOKIEコピー句の読込}

\begin{tabular}{|l|}
\hline
WORKING-STORAGE SECTION.\\
COPY web-cookie.\\
77  rc  pic s9(10).\\
\hline
\end{tabular}

{\gt クッキーのセッティング}

\begin{tabular}{|l|}
\hline
INITIALIZE WEB-COOKIE.
MOVE 'クッキー変数名' TO WEB-NAME.\\
MOVE '変数内容'       TO WEB-VALUE.\\
MOVE '変数の期限'     TO WEB-EXPIRES.\\
MOVE '有効パス'       TO WEB-PATH.\\
MOVE 'ドメイン名'     TO WEB-DOMAIN.\\
MOVE 'SECURE'         TO WEB-SECURE.\\
CALL 'WEB\_SET\_COOKIE' using WEB-COOKIE rc.\\
\hline
\end{tabular}

{\gt 受信クッキーの取得}

\begin{tabular}{|l|}
\hline
CALL 'WEB\_POP\_COOKIE' using WEB\_POP,WEB\_PUSH変数領域名.\\
\hline
\end{tabular}

\item{機能}

クッキーの設定及び取得機能を実現する。
クッキーはブラウザに記憶させる変数であり、クッキー設定を行ったサイトを開く際にブラウザからサーバに送信される物です。

クッキーには有効期限をつけたり、サイトのディレクトリパスを限定する事も可能です。

また、セキュアサイトでのみ有効とする設定も可能です。

クッキーの実際の受信はCALL 'WEB\_GET\_QUERY\_STRING'にて行います。
その後、CALL 'WEB\_POP\_COOKIE'にて変数の値を取り出します。

クッキーの設定ははCALL 'WEB\_SET\_COOKIE'にて行いますが、実際にブラウザに値が送信されるのは

CALL 'WEB\_SHOW'が実行された時と、CALL 'WEB\_SEND\_COOKIE'が実行されたときです。

CALL 'WEB\_SHOW'はSCREENNAMEで指定されるhtmlファイルをWEB\_PUSH,WEB\_POP領域と合成してブラウザへ送信します。

DISPLAYステートメント等によりブラウザ送信用のHTMLファイルを生成する場合はWEB\_SEND\_COOOKIEを使用してください。

\vspace{1em}


\end{itemize}




\subsection{特殊変数}
WEB\_POP,WEB\_PUSH変数領域中に、Web API制御用の特殊変数があります。

この変数をうまく制御する事により、別プログラムのローディングや、外部HTMLファイルを指定する事が出来ます。

現在のところ、この2つの制御用の特殊変数しかありませんが、現状では十分な仕様です。

\subsubsection{PROGRAMNAME}
\begin{itemize}
\item{特殊変数}

  PROGRAMNAME

\item{機能}

CALL 'WEB\_SHOW'実行時に実行したいロードモジュールを指定します。

自ロードモジュールと異なるロードモジュールが指定されたときのみ機能します。

オーバーレイロードされるので、呼び出し元のプログラムに戻ることはありません。

パラメータの受け渡しは環境変数にておこないます。

CALL 'WEB\_PUT\_QUERY\_STRING'を事前に実行しておけば、Web APIグローバル領域の値を環境変数QUERY\_STRINGにセットして新しくロードされるロードモジュールに引き渡す事ができます。


%現在(2003/01/07)は動作しません。

\end{itemize}



\subsubsection{SCREENNAME}
\begin{itemize}
\item{特殊変数}

  SCREENNAME

\item{機能}

CALL 'WEB\_SHOW'実行時に出力したい外部HTMLファイルを指定します。

外部HTMLファイルはWeb APIグローバル変数の値をマージして出力します。

\end{itemize}



\subsection{HTML入力項目の表記方法}
\subsubsection{TEXTボックス (input type=''text'')}
\begin{itemize}

\item{表記方法}

\verb+<input type="text" name="CGI変数名" maxlength="データ長" size="表示サイズ" value="@変数名称">+

\item{説明}

入力フィールドの定義します。

\item{コーディング例}


{\gt html}

\begin{tabular}{|l|}
\hline
\verb+<html lang="ja" dir="ltr">+\\
\verb+<meta http-equiv="content-type" content="text/html;charset=x-euc-jp">+\\
\verb+<form action="test">+\\
\verb+<input type="text" name="testdata" maxlength="80" size="80" value="@testdata">+\\
\verb+<input type="submit" value="実行">+\\
\verb+</form>+\\
\verb+</html>+\\
\hline
\end{tabular}

{\gt cobol} source test.cob

\begin{tabular}{|l|}
\hline
\verb+ working-storage section.+\\
\verb+  01 web-if-rec.+\\
\verb+   02 filler pic x(20) value 'testdata'.+\\
\verb+   02 filler pic x value low-value.+\\
\verb+   02 filler pic x value space.+\\
\verb+   02 filler pic x value low-value+\\.
\verb+   02 testdata pic x(80) value space.+\\
\verb+   02 filler pic x value low-value.+\\
\verb+   02 filler pic x value low-value.+\\
\verb++\\
\verb+ procedure division.+\\
\verb+ main.+\\
\verb+  call 'WEB_QUERY_STRING'.+\\
\verb+  call 'WEB_POP' using web-if-rec.+\\
\verb+* 処理+\\
\verb+  call 'WEB_PUSH' using web-if-rec.+\\
\verb+  call 'WEB_SHOW'.+\\
\verb+  stop.+\\
\hline
\end{tabular}

\end{itemize}


\subsubsection{実行ボタン (input type=''submit'')}
\begin{itemize}

\item{表記方法}

\begin{tabular}{l}
\verb+<input type="submit" name="変数名称" value="CGI変数内容" $変数名称>+\\
\end{tabular}

\item{説明}

実行ボタンを定義します。

\item{コーディング例}

 \begin{enumerate}
 \item{ボタンが一つの場合}

   {\gt html}

   \begin{tabular}{|l|}
   \hline
   \verb+<html lang="ja" dir="ltr">+\\
   \verb+<meta http-equiv="content-type" content="text/html;charset=x-euc-jp">+\\
   \verb+<form action="test">+\\
   \verb+<input type="submit" value="実行">+\\
   \verb+</form>+\\
   \verb+</html>+\\
   \hline
   \end{tabular}

   {\gt cobol} source test.cob

   \begin{tabular}{|l|}
   \hline
   \verb+ working-storage section.+\\
   \verb+  01 web-if-rec.+\\
   \verb+      : : +\\
   \verb+   02 filler pic x value low-value.+\\
   \verb+   02 filler pic x value low-value.+\\
   \verb+ procedure division.+\\
   \verb+ main.+\\
   \verb+  call 'WEB_QUERY_STRING'.+\\
   \verb+  call 'WEB_POP' using web-if-rec.+\\
   \verb+* 処理+\\
   \verb+  call 'WEB_PUSH' using web-if-rec.+\\
   \verb+  call 'WEB_SHOW'.+\\
   \verb+  stop.+\\
   \hline
   \end{tabular}

 \item{ボタンが複数ある場合}

   {\gt html}

   \begin{tabular}{|l|}
   \hline
   \verb+<html lang="ja" dir="ltr">+\\
   \verb+<meta http-equiv="content-type" content="text/html;charset=x-euc-jp">+\\
   \verb+<form action="test">+\\
   \verb+<input type="submit" name="SUBMITMODE" value="計算" $SUBMITMODE>+\\
   \verb+<input type="submit" name="SUBMITMODE" value="検索" $SUBMITMODE>+\\
   \verb+<input type="submit" name="SUBMITMODE" value="保存" $SUBMITMODE>+\\
   \verb+</form>+\\
   \verb+</html>+\\
   \hline
   \end{tabular}

   {\gt cobol} source test.cob

   \begin{tabular}{|l|}
   \hline
   \verb+ working-storage section.+\\
   \verb+  01 web-if-rec.+\\
   \verb+   02 filler pic x(20) value 'SUBMITMODE'. +\\
   \verb+   02 filler pic x value low-value.+\\
   \verb+   02 filler pic x value space.+\\
   \verb+   02 filler pic x value low-value.+\\
   \verb+   02 submitmode pic x (4) value space.+\\
   \verb+   02 filler pic x value low-value.+\\
   \verb+   02 filler pic x value low-value.+\\
   \verb+ procedure division.+\\
   \verb+ main.+\\
   \verb+  call 'WEB_QUERY_STRING'.+\\
   \verb+  call 'WEB_POP' using web-if-rec.+\\
   \verb+  evaluate submitmode of web-if-rec+\\
   \verb+   when '計算'+\\
   \verb+*       計算処理+\\
   \verb+   when '検索'+\\
   \verb+*       検索処理+\\
   \verb+   when '保存'+\\
   \verb+*       保存処理+\\
   \verb+  end-evaluate+\\
   \verb+* 処理+\\
   \verb+  call 'WEB_PUSH' using web-if-rec.+\\
   \verb+  call 'WEB_SHOW'.+\\
   \verb+  stop.+\\
   \hline
   \end{tabular}

 \end{enumerate}
\end{itemize}


\subsubsection{ブルダウンメニュー (select option)}
\begin{itemize}

\item{表記方法}

\begin{tabular}{l}
\verb+<select name="CGI変数名">+\\
\verb+<option value="変数内容" @変数名称:変数内容:SELECTED:>表示コード 表示名称</option>+\\
\verb+<option value="変数内容" @変数名称:変数内容:SELECTED:>表示コード 表示名称</option>+\\
   : : \\
\verb+</select>+\\
\end{tabular}

\item{説明}

プルダウンメニューを定義します。

\item{コーディング例}

{\gt html}

\begin{tabular}{|l|}
\hline
\verb+<HTML LANG="ja" DIR="LTR">+\\
\verb+<META HTTP-EQUIV="Content-Type" CONTENT="text/html; charset=X-EUC-JP">+\\
\verb+<FORM ACTION="test" METHOD="POST">+\\
\verb+血液型<select name="BLOODTYPE">+\\
\verb+<option value="o"  @BLOODTYPE:o:SELECTED:>o  O型 </option>+\\
\verb+<option value="a"  @BLOODTYPE:a:SELECTED:>a  A型 </option>+\\
\verb+<option value="b"  @BLOODTYPE:b:SELECTED:>b  B型 </option>+\\
\verb+<option value="ab" @BLOODTYPE:ab:SELECTED:>ab AB型</option>+\\
\verb+</select>+\\
\verb+<input type="submit" value="実行">+\\
\verb+</form>+\\
\verb+</html>+\\
\hline
\end{tabular}

{\gt cobol} source test.cob

\begin{tabular}{|l|}
\hline
\verb+ working-storage section.+\\
\verb+  01 web-if-rec.+\\
\verb+   02 filler pic x(20) value 'BLOODTYPE'.+\\
\verb+   02 filler pic x value low-value.+\\
\verb+   02 filler pic x value space.+\\
\verb+   02 filler pic x value low-value+\\.
\verb+   02 BLOODTYPE pic x(2) value space.+\\
\verb+   02 filler pic x value low-value.+\\
\verb+   02 filler pic x value low-value.+\\
\verb++\\
\verb+ procedure division.+\\
\verb+ main.+\\
\verb+  call 'WEB_QUERY_STRING'.+\\
\verb+  call 'WEB_POP' using web-if-rec.+\\
\verb+* 処理+\\
\verb+  call 'WEB_PUSH' using web-if-rec.+\\
\verb+  call 'WEB_SHOW'.+\\
\verb+  stop.+\\
\hline
\end{tabular}

\end{itemize}




\subsubsection{TEXTボックス(複数行可能)(textarea)}
\begin{itemize}

\item{表記方法}

\begin{tabular}{l}
\verb+<textarea cols="横幅" rows="縦幅" name="CGI変数名">@変数名</textarea>+\\
\end{tabular}

\item{説明}

複数行入力可能なテキストボックスを定義します。

\item{コーディング例}

{\gt html}

\begin{tabular}{|l|}
\hline
\verb+<html lang="ja" dir="ltr">+\\
\verb+<meta http-equiv="content-type" content="text/html;charset=x-euc-jp">+\\
\verb+<form action="test">+\\
\verb+<textarea name="testdata" cols="80" rows="40">@textdata</textarea>+\\
\verb+<input type="submit" value="実行">+\\
\verb+</form>+\\
\verb+</html>+\\
\hline
\end{tabular}

{\gt cobol} source test.cob

\begin{tabular}{|l|}
\hline
\verb+ working-storage section.+\\
\verb+  01 web-if-rec.+\\
\verb+   02 filler pic x(20) value 'testdata'.+\\
\verb+   02 filler pic x value low-value.+\\
\verb+   02 filler pic x value space.+\\
\verb+   02 filler pic x value low-value+\\.
\verb+   02 testdata pic x(3200) value space.+\\
\verb+   02 filler pic x value low-value.+\\
\verb+   02 filler pic x value low-value.+\\
\verb++\\
\verb+ procedure division.+\\
\verb+ main.+\\
\verb+  call 'WEB_QUERY_STRING'.+\\
\verb+  call 'WEB_POP' using web-if-rec.+\\
\verb+* 処理+\\
\verb+  call 'WEB_PUSH' using web-if-rec.+\\
\verb+  call 'WEB_SHOW'.+\\
\verb+  stop.+\\
\hline
\end{tabular}

\end{itemize}

\newpage
\section{開発環境のセットアップ}

まず、開発環境を整えましょう。プログラミングの醍醐味は動かしてこそ実感できるものです。
ここではWindows\footnote{WindowsはMicrosoftの商標です。ここではWindows95,98,98SE,ME,NT,2000,XPを対象とします。}
及びLinuxでの開発環境構築手順を説明します。

\subsection{Windowsの開発環境をセットアップする}
\subsubsection{Webサーバ AN HTTPDのインストール}

AN HTTPD\footnote{AN HTTPDは中田昭雄氏作成のWebサーバソフトです。}はWindowsで利用できるWebサーバの一つです。

ダウンロードの方法

http://www.st.rim.or.jp/~nakata/よりダウンロード出来ます。
執筆時の最新バージョンはVersion1.42gで、ファイル名はhttpd142g.exeです。

このファイルをダウンロードし、ダブルクリックにて起動してください。

以上でインストールは完了です。

\subsubsection{MinGW,OpenCOBOLのインストール}

MinGWとはMinimalist GNU for Windowsの略であり、GCCによるネイティブな32bit Windowsプログラムを生成するための開発環境です。

MinGWと似たものにCygwinという物がありますが、MinGWで生成したEXEはMinGW用のDLLは不要です。
EXEのみで動作可能なネイティブコードを生成することが可能です。

このあたりの詳細は ビットウォークさんのホームページ

{\it http://members10.tsukaeru.net/bitwalk/index.html →MinGW→MinGWについて}

に詳細の記述があります。

詳細はこちらを参照してみてください。

\vspace{1em}

\begin{itemize}

\item{MinGW,tools,open-cobolをダウンロードする}
\footnote{MinGWはhttp://member10.tsukaeru.net/bitwalk/index.htmlよりCD-Rにて購入する事も可能です。}

http://members10.tsukaeru.net/bitwalk/index.html → 「MinGw」をクリック → 「ダウンロード」をクリック
→ 「mingw\_017.exe」「mingw\_tools\_004.exe」「open-cobol-0.10.1\_011.exe」をそれぞれダウロードします。

\item{mingwのインストール}

ダウンロードした{\it mingw\_017.exe}をダブルクリックします。
後はメッセージに従ってインストールを行います。

\item{mingw\_toolsのインストール}

ダウンロードした{\it mingw\_tools\_004.exe}をダブルクリックします。
後はメッセージに従ってインストールを行います。

\item{open-cobolのインストール}

ダウンロードした{\it open-cobol-0.10.1\_011.exe}をダブルクリックします。
後はメッセージに従ってインストールを行います。

\end{itemize}

\subsubsection{COBOL Webインタフェースのインストール}

当資料で紹介しているCOBOL Webインタフェースをインストールします。
COBOL Webインタフェースはソースモジュールによる配布としております。

{\it http://pulse.homeip.net/webif.html}より{\it web.0.2.tar.gz}を{\it c:¥temp}へダウンロードします。

デスクトップにある{\it Msys}アイコンをクリックします。

起動されたMsysウィンドウで、以下のコマンドを実行します。

\$ mv /c/temp/web.0.2.tar.gz .

\$ tar zxvf web.0.2.tar.gz

\$ make libweb.a

\$ cp -p libweb.a /usr/local/lib

以上でインストールは終了です。

/usr/local/lib/libweb.aは実際にはc:¥msys¥1.0¥local¥lib¥libweb.aにコピーされます。







\subsection{Linuxの開発環境をセットアップする}
\subsubsection{WebサーバApacheの設定}

ほとんどのLinuxディストリビューションではApacheのインストールが完了しているので、
当資料ではインストールについては記述しません。

インストール後に必要な設定のみ記述します。

当資料のプログラム環境を整えるためにはApacheに、CGI実行,ユーザディレクトリ,別名定義モジュールを組み込む必要がある。

\vspace{1em}
{\gt 組み込みが必要なApacheモジュール}

\begin{tabular}{|l|l|}
\hline
モジュール名 & モジュールファイル名 \\
\hline
CGI実行モジュール		&mod\_cgi.so	\\
ユーザディレクトリモジュール	&mod\_userdir.so\\
別名定義モジュール		&mod\_alias.so	\\
\hline
\end{tabular}

実際の設定は以下の様になります。(以下の設定例ではapacheのバージョンを1.3としています。)

FileName:/etc/apache/httpd.conf

\begin{tabular}{|l|}
\hline
\verb+LoadModule cgi_module /usr/lib/apache/1.3/mod_cgi.so+\\
\verb+LoadModule userdir_module /usr/lib/apache/1.3/mod_userdir.so+\\
\verb+LoadModule alias_module /usr/lib/apache/1.3/mod_alias.so+\\
\verb+AddHandler cgi-script .cgi+\\
\verb+<Directory /home/cobol/public_html/cgi-bin>+\\
\verb+Options +ExecCGI+\\
\verb+</Directory>+\\
\hline
\end{tabular}

または

FileName:/usr/local/etc/apache/httpd.conf

\begin{tabular}{|l|}
\hline
\verb+LoadModule cgi_module /usr/local/lib/apache/1.3/mod_cgi.so+\\
\verb+LoadModule userdir_module /usr/local/lib/apache/1.3/mod_userdir.so+\\
\verb+LoadModule alias_module /usr/local/lib/apache/1.3/mod_alias.so+\\
\verb+AddHandler cgi-script .cgi+\\
\verb+<Directory /home/cobol/public_html/cgi-bin>+\\
\verb+Options +ExecCGI+\\
\verb+</Directory>+\\
\hline
\end{tabular}

{\gt ユーザ登録}

COBOL開発用のユーザを登録する必要はありませんが、
当資料の便宜上として{\gt cobol}ユーザを登録します。

以下のコマンドは、root(スーパユーザ)にて行ってください。

\begin{tabular}{|l|}
\hline
\verb&# useradd -m -d /home/cobol cobol&\\
\verb&# passwd cobol&\\
\verb&Enter new UNIX password: &\\
\verb&Retype new UNIX password: &\\
\verb&passwd: password updated successfully&\\
\hline
\end{tabular}
Enter new UNIX password:にcobolユーザのパスワードを入力してください。

Retype new UNIX password:にはEnter new UNIX passwordにて入力したパスワードをもう一度入力してください。

CGIファイルを格納するディレクトリを作成します。

\begin{tabular}{|l|}
\hline
\verb&mkdir -p /home/cobol/public_html/cgi-bin&\\
\verb&chown cobol:cobol /home/cobol/public_html/cgi-bin&\\
\verb&chown cobol:cobol /home/cobol/public_html&\\
\verb&chmod 755 /home/cobol/public_html/cgi-bin&\\
\verb&chmod 755 /home/cobol/public_html&\\
\hline
\end{tabular}

 
\subsubsection{COBOLコンパイラのインストール}

当資料で使用するCOBOLコンパイラOPEN COBOL,TinyCobolのインストール方法を紹介します。

\begin{itemize}

\item{OPEN COBOLのインストール}

OPEN COBOLは{\it http://www.open-cobol.org/}よりダウンロード出来ます。

{\it http://www.open-cobol.org/}より{\it open-cobol-0.10.tar.gz}をダウンロードし、以下の手順でコンパイルします。

\begin{tabular}{|l|}
\hline
\verb+# tar zxvf open-cobol-0.10.tar.gz+\\
\verb+# cd open-cobol-0.10+\\
\verb+# ./configure+\\
\verb+# make+\\
\verb+# su+\\
\verb+Password: スーパユーザのパスワードを入力+\\
\verb+# make install+\\
\hline
\end{tabular}

以下のファイルがインストールされます。

\begin{tabular}{ll}
/usr/local/bin/cobc           &COBOLコンパイラ\\
/usr/local/bin/cobpp          &COBOLプリプロセッサ\\
/usr/local/lib/libcob.a       &COBOLライブラリアーカイブ \\
/usr/local/lib/libcob.la      &COBOLライブラリlibtool用ファイル\\
/usr/local/lib/libcob.so      &COBOL共有ライブラリ\\
/usr/local/lib/libcob.so.0    &COBOL共有ライブラリ\\
/usr/local/lib/libcob.so.0.0.0&COBOL共有ライブラリ\\
\end{tabular}

注意:./configureにてエラーが発生した場合、必要なモジュールを組み込んでください。

\item{TinyCobolのインストール}

TinyCobolは{\it http://tiny-cobol.sourceforge.net/}よりダウンロード出来ます。

{\it http://tiny-cobol.sourceforege.net/}より{\it tinycobol-0.58.tar.gz}をダウンロードし、以下の手順でコンパイルします。

\begin{tabular}{|l|}
\hline
\verb+# tar zxvf tinycobol-0.58.tar.gz+\\
\verb+# cd tinycobl-0.58+\\
\verb+# ./configure+\\
\verb+# make+\\
\verb+# su+\\
\verb+Password: スーパユーザのパスワードを入力+\\
\verb+# make install+\\
\hline
\end{tabular}

以下のファイルがインストールされます。

\begin{tabular}{ll}
/usr/local/bin/htcobol            &COBOLコンパイラ\\
/usr/local/bin/cobpp              &COBOLプリプロセッサ\\
/usr/local/man/man1/htcobol.1     &オンラインマニュアル\\
/usr/local/man/man1/htcobf2f.1    &オンラインマニュアル\\
/usr/local/man/man1/htcobolpp.1   &オンラインマニュアル\\
/usr/local/lib/libhtcobol.a       &COBOLライブラリアーカイブ\\
/usr/local/lib/libhtcobol2.a      &COBOLライブラリアーカイブ\\
\end{tabular}

注意:./configureにてエラーが発生した場合、必要なモジュールを組み込んでください。

\end{itemize}

\subsubsection{COBOL Webインタフェースのインストール}

当資料で紹介しているCOBOL Webインタフェースをインストールします。
COBOL Webインタフェースはソースモジュールによる配布としております。

{\it http://pulse.homeip.net/webif.html}より{\it web.0.3.tar.gz}をダウンロードします。

\$ tar zxvf web.0.3.tar.gz

\$ make libweb.a

\$ make install

以上でインストールは終了です。







\section{その他}
\subsection{処理実行中の再リクエストの遮断}

CGIで処理時間を必要とする処理を行う場合、処理が終了するまで再リクエストを遮断する必要があります。

CGIの場合、再リクエストがなされると前回のリクエストの処理はそのままに(処理中であっても)、
再リクエストに対する処理が起動されます。

なかなかレスポンスが無い場合、オペレータはボタンを連打するなどの反応を示す場合があります。
この場合、永久にレスポンスは返らない事になります。

再リクエストを遮断する方法は多種多様であるが、ここではJAVA Scriptを使用し、
Webブラウザ側で再リクエストを遮断する方法を紹介します。

JAVA Scriptで再リクエストを遮断する場合、処理中に別の画面を表示する手法で行います。

処理概要としては、html上にてリクエスト処理中の画面を非表示モードで定義しておき、
リクエスト時に表示する手順で行います。

具体的には以下の様な内容になります。

\vspace{1em}

{\gt 再リクエストの遮断サンプル}

{\footnotesize
\begin{tabular}{|l|}
\hline
\verb+<html LANG="ja" DIR="LTR">+\\
\verb+<meta http-equiv="Content-Type" CONTENT="text/html; charset=X-EUC-JP">+\\
\verb+<script language="JavaScript">+\\
\verb+submit_block_flag = true;+\\
\verb+function submit_block(){+\\
\verb+   document.getElementById('mainform').style.display = 'none';+\\
\verb+   document.getElementById("message").style.visibility="visible";+\\
\verb+   window.defaultStatus = "要求を送信しています。" +\\
\verb+   return true;+\\
\verb+}+\\
\verb+</script>+\\
\verb+<form action="CGIプログラム名" method="POST" onSubmit="submit_block()">+\\
\verb+<span id=mainform>+\\
\verb+ <input type="text" name="DATA1" maxlength="20" size="25" value="@DATA1">  ←ここに画面内容を記述します。+\\
\verb+ <input type="text" name="DATA2" maxlength="20" size="25" value="@DATA2">+\\
\verb+ <input type="text" name="DATA3" maxlength="20" size="25" value="@DATA3">+\\
\verb+ <input type="text" name="DATA4" maxlength="20" size="25" value="@DATA4">+\\
\verb+ <input type="submit">+\\
\verb+ <input type="reset">+\\
\verb+</span>+\\
\verb+<span id="message" style="visibility:hidden;">+\\
\verb+<h2>処理中です。</h2>    ←ここに処理中に表示する内容を記述します。+\\
\verb+</span>+\\
\verb+</form>+\\
\verb+</html>+\\
\hline
\end{tabular}
}

説明しますと、mainformで本来の画面を定義しておき、messageで処理中の画面を定義しておきます。
messageはstyle="visibility:hidden;"にて非表示にしておきます。

リクエストがなされたとき(OnSubmitイベント発生時)にsubmit\_block()関数が実行され、
mainformが非表示となり、messageが表示される。この時点で再リクエストの遮断が成立することとなります。

後はCGI側のリクエストHTMLで、mainformとmessageの表示非表示関係を制御してやれば良いことになります。

もちろんこの方式ですと、JAVA Scriptが実行可能なWebブラウザでなければなりません。また、spanも解釈可能でなければなりませんので、万能というわけではないかもしれませんが、筆者の環境(IE 5.0,moziila 1.0)では正常に動作しています。



\newpage
\section{サンプル}
\subsection{足し算処理}
{\gt html} addsample.html

{\footnotesize 
\begin{tabular}{|l|}
\hline
\vspace{-0.4em}\verb!<html lang="ja" dir="ltr">!\\
\vspace{-0.4em}\verb!<meta http-equiv="content-type" content="text/html;charset=x-euc-jp">!\\
\vspace{-0.4em}\verb!<form action="addsample.cgi" method="post">!\\
\vspace{-0.4em}\verb!足し算処理のサンプルです。<br><br>!\\
\vspace{-0.4em}\verb!!\\
\vspace{-0.4em}\verb!<input type="text" name="DATA1" maxlength="12" size="15" value="@DATA1">!\\
\vspace{-0.4em}\verb!+!\\
\vspace{-0.4em}\verb!<input type="text" name="DATA2" maxlength="12" size="15" value="@DATA2">!\\
\vspace{-0.4em}\verb!=!\\
\vspace{-0.4em}\verb!@RESULT!\\
\vspace{-0.4em}\verb!<input type="submit" value="実行">!\\
\vspace{-0.4em}\verb!<input type="reset" value="取消">!\\
\vspace{-0.4em}\verb!</form>!\\
\vspace{-0.4em}\verb!</html>!\\
\\
\hline
\end{tabular}
}

\vspace{1em}
{\gt cobol source} addsample.cob

{\footnotesize
\begin{tabular}{|l|}
\hline
\vspace{-0.4em}\verb!       IDENTIFICATION        DIVISION.!\\
\vspace{-0.4em}\verb!       PROGRAM-ID.           addsample.!\\
\vspace{-0.4em}\verb!       AUTHOR.               kazui.!\\
\vspace{-0.4em}\verb!       ENVIRONMENT           DIVISION.!\\
\vspace{-0.4em}\verb!       CONFIGURATION         SECTION.!\\
\vspace{-0.4em}\verb!       INPUT-OUTPUT          SECTION.!\\
\vspace{-0.4em}\verb!       DATA                  DIVISION.!\\
\vspace{-0.4em}\verb!       FILE                  SECTION.!\\
\vspace{-0.4em}\verb!      *!\\
\vspace{-0.4em}\verb!       WORKING-STORAGE SECTION.!\\
\vspace{-0.4em}\verb!       01 webif-rec.!\\
\vspace{-0.4em}\verb!         02 filler pic x(20) value 'DATA1'.!\\
\vspace{-0.4em}\verb!         02 filler pic x value low-value.!\\
\vspace{-0.4em}\verb!         02 filler pic x value space.!\\
\vspace{-0.4em}\verb!         02 filler pic x value low-value.!\\
\vspace{-0.4em}\verb!         02 data1 pic x(12) value space.!\\
\vspace{-0.4em}\verb!         02 filler pic x value low-value.!\\
\vspace{-0.4em}\verb!         02 filler pic x(20) value 'DATA2'.!\\
\vspace{-0.4em}\verb!         02 filler pic x value low-value.!\\
\vspace{-0.4em}\verb!         02 filler pic x value space.!\\
\vspace{-0.4em}\verb!         02 filler pic x value low-value.!\\
\vspace{-0.4em}\verb!         02 data2 pic x(12) value space.!\\
\vspace{-0.4em}\verb!         02 filler pic x value low-value.!\\
\vspace{-0.4em}\verb!         02 filler pic x(20) value 'RESULT'.!\\
\vspace{-0.4em}\verb!         02 filler pic x value low-value.!\\
\vspace{-0.4em}\verb!         02 filler pic x value space.!\\
\vspace{-0.4em}\verb!         02 filler pic x value low-value.!\\
\vspace{-0.4em}\verb!         02 result pic x(12) value space.!\\
\vspace{-0.4em}\verb!         02 filler pic x value low-value.!\\
\vspace{-0.4em}\verb!         02 filler pic x value low-value.!\\
\vspace{-0.4em}\verb!       01  webif-in.!\\
\vspace{-0.4em}\verb!         02 data1  pic 9(12).!\\
\vspace{-0.4em}\verb!         02 data2  pic 9(12).!\\
\vspace{-0.4em}\verb!         02 result pic 9(12).!\\
\vspace{-0.4em}\verb!       01  webif-ot.!\\
\vspace{-0.4em}\verb!         02 data1  pic zzzzzzzzzzz9.!\\
\vspace{-0.4em}\verb!         02 data2  pic zzzzzzzzzzz9.!\\
\vspace{-0.4em}\verb!         02 result pic zzzzzzzzzzz9.!\\
\vspace{-0.4em}\verb!      *------------------------------------------------!\\
\vspace{-0.4em}\verb!       PROCEDURE             DIVISION.!\\
\vspace{-0.4em}\verb!       MAIN-PROC             SECTION.!\\
\vspace{-0.4em}\verb!       MAIN-PROC-1.!\\
\vspace{-0.4em}\verb!      *!\\
\vspace{-0.4em}\verb!          call 'WEB_GET_QUERY_STRING'.!\\
\vspace{-0.4em}\verb!          call 'WEB_POP' using webif-rec.!\\
\vspace{-0.4em}\verb!          move corr webif-rec to webif-in.!\\
\vspace{-0.4em}\verb!          compute result of webif-in = data1 of webif-in +!\\
\vspace{-0.4em}\verb!                                       data2 of webif-in .!\\
\vspace{-0.4em}\verb!          move corr webif-in to webif-ot.!\\
\vspace{-0.4em}\verb!          move corr webif-ot to webif-rec.!\\
\vspace{-0.4em}\verb!          call 'WEB_PUSH' using webif-rec.!\\
\vspace{-0.4em}\verb!          call 'WEB_SHOW'.!\\
\vspace{-0.4em}\verb!      *!\\
\vspace{-0.4em}\verb!          STOP RUN.!\\
\vspace{-0.4em}\verb!!\\
\\
\hline
\end{tabular}
}

{\gt メニューHTML(起動用のHTML定義)}

\begin{tabular}{|l|}
\hline
\verb+<html lang="ja" dir="ltr">+\\
\verb+<meta http-equiv="content-type" content="text/html;charset=x-euc-jp">+\\
\verb+<a href="/cgi-bin/addsample?screenname=addsample.html">足し算サンプル</a>+\\
\verb+</html>+\\
\hline
\end{tabular}
\newpage
\subsection{四則演算処理}
{\gt html} calcsample.html

{\footnotesize 
\begin{tabular}{|l|}
\hline
\vspace{-0.4em}\verb!<html lang="ja" dir="ltr">!\\
\vspace{-0.4em}\verb!<meta http-equiv="content-type" content="text/html;charset=x-euc-jp">!\\
\vspace{-0.4em}\verb!<FORM action="calcsample.cgi" METHOD="POST">!\\
\vspace{-0.4em}\verb!四則演算処理のサンプルです。<br><br>!\\
\vspace{-0.4em}\verb!<input type="text" name="DATA1" maxlength="12" size="15" value="@DATA1">!\\
\vspace{-0.4em}\verb!<select name="WMODE">!\\
\vspace{-0.4em}\verb! <option value="a" @WMODE:a:SELECTED:>+  +</option>!\\
\vspace{-0.4em}\verb! <option value="s" @WMODE:s:SELECTED:>-  −</option>!\\
\vspace{-0.4em}\verb! <option value="m" @WMODE:m:SELECTED:>*  ×</option>!\\
\vspace{-0.4em}\verb! <option value="d" @WMODE:d:SELECTED:>/  ÷</option>!\\
\vspace{-0.4em}\verb!</select>!\\
\vspace{-0.4em}\verb!<input type="text" name="DATA2" maxlength="12" size="15" value="@DATA2">=@RESULT!\\
\vspace{-0.4em}\verb!<input type="submit" value="実行">!\\
\vspace{-0.4em}\verb!<input type="reset" value="取消">!\\
\vspace{-0.4em}\verb!@あいうえお!\\
\vspace{-0.4em}\verb!</FORM>!\\
\vspace{-0.4em}\verb!</html>!\\
\\
\hline
\end{tabular}
}

\vspace{1em}
{\gt cobol source} calcsample.cob

{\footnotesize
\begin{tabular}{|l|}
\hline
\vspace{-0.4em}\verb!       IDENTIFICATION        DIVISION.!\\
\vspace{-0.4em}\verb!       PROGRAM-ID.           calcsample.!\\
\vspace{-0.4em}\verb!       AUTHOR.               kazui.!\\
\vspace{-0.4em}\verb!       ENVIRONMENT           DIVISION.!\\
\vspace{-0.4em}\verb!       CONFIGURATION         SECTION.!\\
\vspace{-0.4em}\verb!       INPUT-OUTPUT          SECTION.!\\
\vspace{-0.4em}\verb!       DATA                  DIVISION.!\\
\vspace{-0.4em}\verb!       WORKING-STORAGE SECTION. !\\
\vspace{-0.4em}\verb!       01 webif-rec.!\\
\vspace{-0.4em}\verb!         02 filler pic x(20) value 'WMODE'.!\\
\vspace{-0.4em}\verb!         02 filler pic x value low-value.!\\
\vspace{-0.4em}\verb!         02 filler pic x value space.!\\
\vspace{-0.4em}\verb!         02 filler pic x value low-value.!\\
\vspace{-0.4em}\verb!         02 wmode  pic x value space.!\\
\vspace{-0.4em}\verb!         02 filler pic x value low-value.!\\
\vspace{-0.4em}\verb!         02 filler pic x(20) value 'DATA1'.!\\
\vspace{-0.4em}\verb!         02 filler pic x value low-value.!\\
\vspace{-0.4em}\verb!         02 filler pic x value space.!\\
\vspace{-0.4em}\verb!         02 filler pic x value low-value.!\\
\vspace{-0.4em}\verb!         02 data1  pic x(12) value space.!\\
\vspace{-0.4em}\verb!         02 filler pic x value low-value.!\\
\vspace{-0.4em}\verb!         02 filler pic x(20) value 'DATA2'.!\\
\vspace{-0.4em}\verb!         02 filler pic x value low-value.!\\
\vspace{-0.4em}\verb!         02 filler pic x value space.!\\
\vspace{-0.4em}\verb!         02 filler pic x value low-value.!\\
\vspace{-0.4em}\verb!         02 data2 pic x(12) value space.!\\
\vspace{-0.4em}\verb!         02 filler pic x value low-value.!\\
\vspace{-0.4em}\verb!         02 filler pic x(20) value 'RESULT'.!\\
\vspace{-0.4em}\verb!         02 filler pic x value low-value.!\\
\vspace{-0.4em}\verb!         02 filler pic x value space.!\\
\vspace{-0.4em}\verb!         02 filler pic x value low-value.!\\
\vspace{-0.4em}\verb!         02 result pic x(13) value space.!\\
\vspace{-0.4em}\verb!         02 filler pic x value low-value.!\\
\vspace{-0.4em}\verb!         02 filler pic x value low-value.!\\
\vspace{-0.4em}\verb!       01  webif-in.!\\
\vspace{-0.4em}\verb!         02 wmode  pic x.!\\
\vspace{-0.4em}\verb!         02 data1  pic 9(12).!\\
\vspace{-0.4em}\verb!         02 data2  pic s9(12).!\\
\vspace{-0.4em}\verb!         02 result pic s9(12).!\\
\vspace{-0.4em}\verb!       01  webif-ot.!\\
\vspace{-0.4em}\verb!         02 wmode  pic x.!\\
\vspace{-0.4em}\verb!         02 data1  pic zzzzzzzzzzz9.!\\
\vspace{-0.4em}\verb!         02 data2  pic zzzzzzzzzzz9.!\\
\vspace{-0.4em}\verb!         02 result pic -zzzzzzzzzzz9.!\\
\vspace{-0.4em}\verb!       77 w-x pic x.!\\
\vspace{-0.4em}\verb!       PROCEDURE             DIVISION.!\\
\vspace{-0.4em}\verb!       MAIN-PROC             SECTION.!\\
\vspace{-0.4em}\verb!       MAIN-PROC-1.!\\
\vspace{-0.4em}\verb!          call 'WEB_GET_QUERY_STRING'.!\\
\vspace{-0.4em}\verb!          call 'WEB_POP' using webif-rec.!\\
\vspace{-0.4em}\verb!          move corr webif-rec to webif-in.!\\
\vspace{-0.4em}\verb!          evaluate wmode of webif-in!\\
\vspace{-0.4em}\verb!           when 'a'!\\
\vspace{-0.4em}\verb!            compute result of webif-in = data1 of webif-in +!\\
\vspace{-0.4em}\verb!                                       data2 of webif-in!\\
\vspace{-0.4em}\verb!           when 's'!\\
\vspace{-0.4em}\verb!            compute result of webif-in = data1 of webif-in -!\\
\vspace{-0.4em}\verb!                                       data2 of webif-in!\\
\vspace{-0.4em}\verb!           when 'm'!\\
\vspace{-0.4em}\verb!            compute result of webif-in = data1 of webif-in *!\\
\vspace{-0.4em}\verb!                                       data2 of webif-in!\\
\vspace{-0.4em}\verb!           when 'd'!\\
\vspace{-0.4em}\verb!            if data2 of webif-in = zero then!\\
\vspace{-0.4em}\verb!              move zero to result of webif-in!\\
\vspace{-0.4em}\verb!             else!\\
\vspace{-0.4em}\verb!            compute result of webif-in = data1 of webif-in /!\\
\vspace{-0.4em}\verb!                                       data2 of webif-in!\\
\vspace{-0.4em}\verb!            end-if!\\
\vspace{-0.4em}\verb!          end-evaluate.!\\
\vspace{-0.4em}\verb!          move corr webif-in to webif-ot.!\\
\vspace{-0.4em}\verb!          move corr webif-ot to webif-rec.!\\
\vspace{-0.4em}\verb!          call 'WEB_PUSH' using webif-rec.!\\
\vspace{-0.4em}\verb!          call 'WEB_SHOW'.!\\
\vspace{-0.4em}\verb!          STOP RUN.!\\
\vspace{-0.4em}\verb!!\\
\vspace{-0.4em}\verb!!\\
\vspace{-0.4em}\verb!!\\
\vspace{-0.4em}\verb!!\\
\vspace{-0.4em}\verb!!\\
\vspace{-0.4em}\verb!!\\
\\
\hline
\end{tabular}
}

{\gt メニューHTML(起動用のHTML定義)}

\begin{tabular}{|l|}
\hline
\verb+<html lang="ja" dir="ltr">+\\
\verb+<meta http-equiv="content-type" content="text/html;charset=x-euc-jp">+\\
\verb+<a href="/cgi-bin/calcsample?screenname=calcsample.html">四則演算サンプル</a>+\\
\verb+</html>+\\
\hline
\end{tabular}
\newpage
\subsection{クッキー}
{\gt html} cookie.html

{\footnotesize 
\begin{tabular}{|l|}
\hline
\vspace{-0.4em}\verb!<html lang="ja" dir="ltr">!\\
\vspace{-0.4em}\verb!<meta http-equiv="content-type" content="text/html"; charset=X-EUC-JP">!\\
\vspace{-0.4em}\verb!!\\
\vspace{-0.4em}\verb!クッキーサンプル あっほー!\\
\vspace{-0.4em}\verb!<FORM action="cookie.cgi" METHOD="POST">!\\
\vspace{-0.4em}\verb!!\\
\vspace{-0.4em}\verb!<input type="text" name="data1" maxlength="12" size="15" value="@DATA1">!\\
\vspace{-0.4em}\verb!<input type="submit" value="実行">!\\
\vspace{-0.4em}\verb!<input type="reset" value="取消">!\\
\vspace{-0.4em}\verb!</form>!\\
\vspace{-0.4em}\verb!</html>!\\
\\
\hline
\end{tabular}
}

\vspace{1em}
{\gt cobol source} cookie.cob

{\footnotesize
\begin{tabular}{|l|}
\hline
\vspace{-0.4em}\verb!       IDENTIFICATION        DIVISION.!\\
\vspace{-0.4em}\verb!       PROGRAM-ID.           cookie.!\\
\vspace{-0.4em}\verb!       AUTHOR.               kazui.!\\
\vspace{-0.4em}\verb!       ENVIRONMENT           DIVISION.!\\
\vspace{-0.4em}\verb!       CONFIGURATION         SECTION.!\\
\vspace{-0.4em}\verb!       INPUT-OUTPUT          SECTION.!\\
\vspace{-0.4em}\verb!       DATA                  DIVISION.!\\
\vspace{-0.4em}\verb!       FILE                  SECTION.!\\
\vspace{-0.4em}\verb!      *!\\
\vspace{-0.4em}\verb!       WORKING-STORAGE SECTION.!\\
\vspace{-0.4em}\verb!       01 http-cookie-rec.!\\
\vspace{-0.4em}\verb!        02 filler  pic x(10) value 'DATA1'.!\\
\vspace{-0.4em}\verb!        02 filler  pic x value low-value.!\\
\vspace{-0.4em}\verb!        02 filler pic x value space.!\\
\vspace{-0.4em}\verb!        02 filler pic x value low-value.!\\
\vspace{-0.4em}\verb!        02 data1  pic x(16) value space.!\\
\vspace{-0.4em}\verb!        02 filler pic x value low-value.!\\
\vspace{-0.4em}\verb!        02 filler pic x value low-value.!\\
\vspace{-0.4em}\verb!!\\
\vspace{-0.4em}\verb!       01 web-if-rec.!\\
\vspace{-0.4em}\verb!        02 filler pic x(15) value 'SCREENNAME'.!\\
\vspace{-0.4em}\verb!        02 filler pic x value low-value.!\\
\vspace{-0.4em}\verb!        02 filler pic x value space.!\\
\vspace{-0.4em}\verb!        02 filler pic x value low-value.!\\
\vspace{-0.4em}\verb!        02 wscreenname pic x(80) value 'cookie.html'.!\\
\vspace{-0.4em}\verb!        02 filler pic x value low-value.!\\
\vspace{-0.4em}\verb!        02 filler pic x(10) value 'DATA1'.!\\
\vspace{-0.4em}\verb!        02 filler pic x value low-value.!\\
\vspace{-0.4em}\verb!        02 filler pic x value space.!\\
\vspace{-0.4em}\verb!        02 filler pic x value low-value.!\\
\vspace{-0.4em}\verb!        02 data1  pic x(16) value space.!\\
\vspace{-0.4em}\verb!        02 filler pic x value low-value.!\\
\vspace{-0.4em}\verb!        02 filler pic x value low-value.!\\
\vspace{-0.4em}\verb!!\\
\vspace{-0.4em}\verb!       COPY web-cookie.!\\
\vspace{-0.4em}\verb!       01 web-rc   pic 9(10) binary.!\\
\vspace{-0.4em}\verb!      *------------------------------------------------!\\
\vspace{-0.4em}\verb!       PROCEDURE             DIVISION.!\\
\vspace{-0.4em}\verb!       MAIN-PROC             SECTION.!\\
\vspace{-0.4em}\verb!       MAIN-PROC-1.!\\
\vspace{-0.4em}\verb!      *!\\
\vspace{-0.4em}\verb!          call 'WEB_GET_QUERY_STRING'.!\\
\vspace{-0.4em}\verb!          call 'WEB_POP'        using web-if-rec.!\\
\vspace{-0.4em}\verb!          move 'cookie.html' to wscreenname of web-if-rec.!\\
\vspace{-0.4em}\verb!          if data1 of web-if-rec  = SPACE then!\\
\vspace{-0.4em}\verb!           call 'WEB_POP_COOKIE' using http-cookie-rec!\\
\vspace{-0.4em}\verb!           move data1 of http-cookie-rec to data1 of web-if-rec!\\
\vspace{-0.4em}\verb!          else!\\
\vspace{-0.4em}\verb!           initialize WEB-COOKIE!\\
\vspace{-0.4em}\verb!           move 'DATA1'   to WEB-NAME!\\
\vspace{-0.4em}\verb!           move  data1 of web-if-rec to WEB-VALUE!\\
\vspace{-0.4em}\verb!           call  'WEB_SET_COOKIE' using WEB-COOKIE WEB-RC!\\
\vspace{-0.4em}\verb!!\\
\vspace{-0.4em}\verb!           initialize WEB-COOKIE!\\
\vspace{-0.4em}\verb!           move 'DATA2'   to WEB-NAME!\\
\vspace{-0.4em}\verb!           move  'test save' to WEB-VALUE!\\
\vspace{-0.4em}\verb!           call  'WEB_SET_COOKIE' using WEB-COOKIE WEB-RC!\\
\vspace{-0.4em}\verb!          end-if.!\\
\vspace{-0.4em}\verb!          call 'WEB_PUSH' using web-if-rec.!\\
\vspace{-0.4em}\verb!!\\
\vspace{-0.4em}\verb!!\\
\vspace{-0.4em}\verb!          call 'WEB_SHOW'!\\
\vspace{-0.4em}\verb!!\\
\vspace{-0.4em}\verb!      *!\\
\vspace{-0.4em}\verb!          STOP RUN.!\\
\vspace{-0.4em}\verb!!\\
\vspace{-0.4em}\verb!!\\
\\
\hline
\end{tabular}
}

{\gt メニューHTML(起動用のHTML定義)}

\begin{tabular}{|l|}
\hline
\verb+<html lang="ja" dir="ltr">+\\
\verb+<meta http-equiv="content-type" content="text/html;charset=x-euc-jp">+\\
\verb+<a href="/cgi-bin/cookie.cgi">クッキーサンプル</a>+\\
\verb+</html>+\\
\hline
\end{tabular}

\newpage
\subsection{チャット}
{\gt html} chat.html

{\footnotesize 
\begin{verbatim}
<!doctype html public "-//W3C//DTD HTML 4.01 Frameset//EN" "http://www.w3.org/TR/html4/frameset.dtd">
<html lang="ja" dir="ltr">
<meta http-equiv="content-type" content="text/html"; charset=X-EUC-JP">
<head>
<title>Open COBOL Chat Program</title>
</head>

<frameset rows="15%,*" >
 <frame src="cgi-bin/chat.cgi?screenname=chatlogin.html" name="up" frameborder="1">
 <frame name="down"     frameborder="1">
</frameset>
</html>
\end{verbatim}

\vspace{1em}
{\gt cobol source} chat.cob

{\footnotesize
\begin{verbatim}
       IDENTIFICATION        DIVISION.
       PROGRAM-ID.           chat.
       AUTHOR.               ys62.
       ENVIRONMENT           DIVISION.
       CONFIGURATION         SECTION.
       INPUT-OUTPUT          SECTION.
       FILE-CONTROL.
        select chat-file assign to "./chat.dat"
          organization line sequential access sequential
          file status is chat-file-fs.
       DATA                  DIVISION.
       FILE                  SECTION.
       fd chat-file.
       01 chat-rec.
          02 wcolor   pic x(30).
          02 wname    pic x(30).
          02 wtime    pic x(20).
          02 wdata    pic x(200).
      *
       WORKING-STORAGE SECTION.
       77 chat-file-fs     pic 9(02).

       01 wchat-rec.
          02 wchat-rec-tab occurs 40.
           03 wcolor   pic x(30).
           03 wname    pic x(30).
           03 wdata    pic x(200).
           03 wtime    pic x(20).
         
       copy chat-param.

       copy web-cookie.

       01 http-cookie.
         02 filler pic x(10) value 'NAME'.
         02 filler pic x value low-value.
         02 filler pic x value space.
         02 filler pic x value low-value.
         02 wname  pic x(30) value space.
         02 filler pic x value low-value.
         02 filler pic x(10) value 'COLOR'.
         02 filler pic x value low-value.
         02 filler pic x value space.
         02 filler pic x value low-value.
         02 wcolor pic x(30) value space.
         02 filler pic x value low-value.
         02 filler pic x value low-value.
      
       77 ret-value       pic s9(9).
      
       77 w-x             pic x(20).
      
       77 i               pic 9(2).
      *
       77 wwdate           pic x(6).
       77 wwtime           pic x(8).
       77 chat-rec-eof     pic x(3) value space.
       01 arg-command-line.
        02 arg-command-line-d pic x(160) value space.
        02 filler             pic x      value zero.
       77 rc pic s9(10) usage binary.
      *----------------------------------------------------------------
       PROCEDURE             DIVISION.
       MAIN-PROC             SECTION.
       MAIN-PROC-1.
      *
      *   accept arg-command-line-d from command-line.
          call 'WEB_GET_QUERY_STRING'.
          call 'WEB_POP' using chat-param-if.
          call 'WEB_POP_COOKIE' using http-cookie.
          move 1 to i.

          accept wwdate from date.
          accept wwtime from time.
      *   ハンドル名がなければログインメッセージを出して終了
          if wname of chat-param-if = SPACE then
             move 'chatlogin.html' to wscreenname of chat-param-if
             move 'up'             to wtarget     of chat-param-if
             move wcolor of chat-rec to wcolor of wchat-rec-tab (i)
             move wname  of chat-rec to wname  of wchat-rec-tab (i)
             move wname  of http-cookie to wname  of chat-param-if
             move wcolor of http-cookie to wcolor of chat-param-if
             perform SET-COOKIE
             call 'WEB_PUSH' using chat-param-if
             call 'WEB_SHOW'
             go to 99
          end-if.
          if wname of chat-param-if not = SPACE and
             wscreenname of chat-param-if = 'chatlogin.html' then
             move 'chatsend.html'   to wscreenname of chat-param-if
             move 'down'        to wtarget     of chat-param-if
             PERFORM SET-COOKIE
             call 'WEB_PUSH' using chat-param-if
             call 'WEB_SHOW'
             go to 99
          end-if.
      *
          PERFORM SET-COOKIE.

      *   move 'chatsend.html'   to wscreenname of chat-param-if.
      *   move 'down'        to wtarget     of chat-param-if.
          move 'chatlog.html' to wscreenname of chat-param-if.
          call 'WEB_PUSH' using chat-param-if.
          perform SHOW-LOG.
          call 'WEB_SHOW'.
      *    go to 99.          
      
       99.
          stop run.
      *
       SET-COOKIE SECTION.
       00.
          initialize  web-cookie.
          move 'NAME'                 to web-name.
          move wname of chat-param-if to web-value.
          move 'Fri, 31-Dec-2030 23:59:59' to web-expires.
          call 'WEB_SET_COOKIE' using WEB-COOKIE RC.

          initialize  web-cookie.
          move 'COLOR'                 to web-name.
          move wcolor of chat-param-if to web-value.
          move 'Fri, 31-Dec-2030 23:59:59' to web-expires.
          call 'WEB_SET_COOKIE' using WEB-COOKIE RC.
       99.
          EXIT.

       SHOW-LOG section.
          move corr chat-param-if to chat-param-in.
          initialize wchat-rec.
          move space to chat-rec-eof.
          open  input chat-file.
          if chat-file-fs = 0 then
            read chat-file 
            at end
              move 'EOF' to chat-rec-eof
            end-read
            move zero to i
            perform until chat-rec-eof = 'EOF'
              add 1 to i
              move wcolor of chat-rec to wcolor of wchat-rec-tab (i)
              move wname  of chat-rec to wname  of wchat-rec-tab (i)
              move wdata  of chat-rec to wdata  of wchat-rec-tab (i)
              move wtime  of chat-rec to wtime  of wchat-rec-tab (i)
              read chat-file
               at end 
                 move 'EOF' to chat-rec-eof
              end-read
            end-perform
            move wchat-rec to dat-table of chat-param-in-tab
          end-if.
          close chat-file.
      
          if wsenddata of chat-param-in not = space 
            and wmode  of chat-param-in not = "Reload" then
           perform varying i from 40 by -1 until i <= 1
        
            move wname of dat-table-item of chat-param-in-tab ( i - 1 ) 
              to wname of dat-table-item of chat-param-in-tab (i)
            move wdata of dat-table-item of chat-param-in-tab ( i - 1 ) 
              to wdata of dat-table-item of chat-param-in-tab (i)
            move wtime of dat-table-item of chat-param-in-tab ( i - 1 ) 
              to wtime of dat-table-item of chat-param-in-tab (i)
            move wcolor of dat-table-item of chat-param-in-tab( i - 1 ) 
              to wcolor of dat-table-item of chat-param-in-tab (i)
           end-perform
           move wsenddata of chat-param-in-tab to
           wdata of dat-table-item of dat-table of chat-param-in-tab(1)
           move wname     of chat-param-in to 
            wname of dat-table-item of chat-param-in-tab (1)
           move wcolor   of chat-param-in to 
            wcolor of dat-table-item of chat-param-in-tab (1)
           string 
              wwdate(1:2) delimited size 
              '/'         delimited size
              wwdate(3:2) delimited size 
              '/'         delimited size
              wwdate(5:2) delimited size 
              ' '         delimited size
              wwtime(1:2) delimited size 
              ':'         delimited size
              wwtime(3:2) delimited size 
              ':'         delimited size
              wwtime(5:2) delimited size 
            into wtime of dat-table-item of chat-param-in-tab (1)
           end-string
           open output chat-file
           perform varying i from 1 by 1 until i > 40
             move wcolor of dat-table of chat-param-in-tab (i) 
               to wcolor of chat-rec
             move wname  of dat-table of chat-param-in-tab (i) 
               to wname  of chat-rec
             move wcolor of dat-table of chat-param-in-tab (i) 
               to wcolor of chat-rec
             move wdata  of dat-table of chat-param-in-tab (i) 
               to wdata  of chat-rec
             move wtime  of dat-table of chat-param-in-tab (i) 
               to wtime  of chat-rec
             write chat-rec
           end-perform
           close chat-file
           move space to wsenddata of chat-param-in
          end-if.
      
          move corr chat-param-in  to chat-param-ot.
          move corr chat-param-ot  to chat-param-if.
      
          call 'WEB_PUSH' using chat-param-if.
      *   call 'WEB_SHOW'.
       99.
          exit.
\end{verbatim}
{\gt copy file} chat-param

{\footnotesize
\begin{verbatim}
       01 chat-param-if.
      *
          02 filler pic x(10) value "NAME".
          02 filler pic x value low-value.
          02 filler pic x value space.
          02 filler pic x value low-value.
          02 wname pic x(30) value space.
          02 filler pic x value low-value.
      *    
          02 filler pic x(10) value "SENDDATA".
          02 filler pic x value low-value.
          02 filler pic x value space.
          02 filler pic x value low-value.
          02 wsenddata pic x(200) value space.
          02 filler pic x value low-value.
      *    
          02 filler pic x(10) value "MESSAGE1".
          02 filler pic x value low-value.
          02 filler pic x value space.
          02 filler pic x value low-value.
          02 wmessage1 pic x(80) value space.
          02 filler pic x value low-value.
      *    
          02 filler pic x(10) value "MESSAGE2".
          02 filler pic x value low-value.
          02 filler pic x value space.
          02 filler pic x value low-value.
          02 wmessage2 pic x(80) value space.
          02 filler pic x value low-value.
      *    
          02 filler pic x(10) value "MESSAGE3".
          02 filler pic x value low-value.
          02 filler pic x value space.
          02 filler pic x value low-value.
          02 wmessage3 pic x(80) value space.
          02 filler pic x value low-value.
      *    
          02 filler pic x(10) value "TARGET".
          02 filler pic x value low-value.
          02 filler pic x value space.
          02 filler pic x value low-value.
          02 wtarget pic x(80) value space.
          02 filler pic x value low-value.
      *   
          02 filler pic x(10) value "COLOR".
          02 filler pic x value low-value.
          02 filler pic x value space.
          02 filler pic x value low-value.
          02 wcolor pic x(30) value space.
          02 filler pic x value low-value.
      *    
          02 filler pic x(10) value "COLOR1".
          02 filler pic x value low-value.
          02 filler pic x value space.
          02 filler pic x value low-value.
          02 wcolor1 pic x(30) value space.
          02 filler pic x value low-value.
      *    
          02 filler pic x(10) value "NAME1".
          02 filler pic x value low-value.
          02 filler pic x value space.
          02 filler pic x value low-value.
          02 wname1 pic x(30) value space.
          02 filler pic x value low-value.
      *    
          02 filler pic x(10) value "DATA1".
          02 filler pic x value low-value.
          02 filler pic x value space.
          02 filler pic x value low-value.
          02 wdata1 pic x(200) value space.
          02 filler pic x value low-value.
      *    
          02 filler pic x(10) value "TIME1".
          02 filler pic x value low-value.
          02 filler pic x value space.
          02 filler pic x value low-value.
          02 wtime1 pic x(20) value space.
          02 filler pic x value low-value.
      *    
          02 filler pic x(10) value "COLOR2".
          02 filler pic x value low-value.
          02 filler pic x value space.
          02 filler pic x value low-value.
          02 wcolor2 pic x(30) value space.
          02 filler pic x value low-value.
      *    
          02 filler pic x(10) value "NAME2".
          02 filler pic x value low-value.
          02 filler pic x value space.
          02 filler pic x value low-value.
          02 wname2 pic x(30) value space.
          02 filler pic x value low-value.
      *    
          02 filler pic x(10) value "DATA2".
          02 filler pic x value low-value.
          02 filler pic x value space.
          02 filler pic x value low-value.
          02 wdata2 pic x(200) value space.
          02 filler pic x value low-value.
      *    
          02 filler pic x(10) value "TIME2".
          02 filler pic x value low-value.
          02 filler pic x value space.
          02 filler pic x value low-value.
          02 wtime2 pic x(20) value space.
          02 filler pic x value low-value.
      *    
          02 filler pic x(10) value "COLOR3".
          02 filler pic x value low-value.
          02 filler pic x value space.
          02 filler pic x value low-value.
          02 wcolor3 pic x(30) value space.
          02 filler pic x value low-value.
      *    
          02 filler pic x(10) value "NAME3".
          02 filler pic x value low-value.
          02 filler pic x value space.
          02 filler pic x value low-value.
          02 wname3 pic x(30) value space.
          02 filler pic x value low-value.
      *    
          02 filler pic x(10) value "DATA3".
          02 filler pic x value low-value.
          02 filler pic x value space.
          02 filler pic x value low-value.
          02 wdata3 pic x(200) value space.
          02 filler pic x value low-value.
      *    
          02 filler pic x(10) value "TIME3".
          02 filler pic x value low-value.
          02 filler pic x value space.
          02 filler pic x value low-value.
          02 wtime3 pic x(20) value space.
          02 filler pic x value low-value.
      *    
          02 filler pic x(10) value "COLOR4".
          02 filler pic x value low-value.
          02 filler pic x value space.
          02 filler pic x value low-value.
          02 wcolor4 pic x(30) value space.
          02 filler pic x value low-value.
      *    
          02 filler pic x(10) value "NAME4".
          02 filler pic x value low-value.
          02 filler pic x value space.
          02 filler pic x value low-value.
          02 wname4 pic x(30) value space.
          02 filler pic x value low-value.
      *    
          02 filler pic x(10) value "DATA4".
          02 filler pic x value low-value.
          02 filler pic x value space.
          02 filler pic x value low-value.
          02 wdata4 pic x(200) value space.
          02 filler pic x value low-value.
      *    
          02 filler pic x(10) value "TIME4".
          02 filler pic x value low-value.
          02 filler pic x value space.
          02 filler pic x value low-value.
          02 wtime4 pic x(20) value space.
          02 filler pic x value low-value.
      *    
          02 filler pic x(10) value "COLOR5".
          02 filler pic x value low-value.
          02 filler pic x value space.
          02 filler pic x value low-value.
          02 wcolor5 pic x(30) value space.
          02 filler pic x value low-value.
      *    
          02 filler pic x(10) value "NAME5".
          02 filler pic x value low-value.
          02 filler pic x value space.
          02 filler pic x value low-value.
          02 wname5 pic x(30) value space.
          02 filler pic x value low-value.
      *    
          02 filler pic x(10) value "DATA5".
          02 filler pic x value low-value.
          02 filler pic x value space.
          02 filler pic x value low-value.
          02 wdata5 pic x(200) value space.
          02 filler pic x value low-value.
      *    
          02 filler pic x(10) value "TIME5".
          02 filler pic x value low-value.
          02 filler pic x value space.
          02 filler pic x value low-value.
          02 wtime5 pic x(200) value space.
          02 filler pic x value low-value.
      *    
          02 filler pic x(10) value "COLOR6".
          02 filler pic x value low-value.
          02 filler pic x value space.
          02 filler pic x value low-value.
          02 wcolor6 pic x(30) value space.
          02 filler pic x value low-value.
      *    
          02 filler pic x(10) value "NAME6".
          02 filler pic x value low-value.
          02 filler pic x value space.
          02 filler pic x value low-value.
          02 wname6 pic x(30) value space.
          02 filler pic x value low-value.
      *    
          02 filler pic x(10) value "DATA6".
          02 filler pic x value low-value.
          02 filler pic x value space.
          02 filler pic x value low-value.
          02 wdata6 pic x(200) value space.
          02 filler pic x value low-value.
      *    
          02 filler pic x(10) value "TIME6".
          02 filler pic x value low-value.
          02 filler pic x value space.
          02 filler pic x value low-value.
          02 wtime6 pic x(20) value space.
          02 filler pic x value low-value.
      *    
          02 filler pic x(10) value "COLOR7".
          02 filler pic x value low-value.
          02 filler pic x value space.
          02 filler pic x value low-value.
          02 wcolor7 pic x(30) value space.
          02 filler pic x value low-value.
      *    
          02 filler pic x(10) value "NAME7".
          02 filler pic x value low-value.
          02 filler pic x value space.
          02 filler pic x value low-value.
          02 wname7 pic x(30) value space.
          02 filler pic x value low-value.
      *    
          02 filler pic x(10) value "DATA7".
          02 filler pic x value low-value.
          02 filler pic x value space.
          02 filler pic x value low-value.
          02 wdata7 pic x(200) value space.
          02 filler pic x value low-value.
      *    
          02 filler pic x(10) value "TIME7".
          02 filler pic x value low-value.
          02 filler pic x value space.
          02 filler pic x value low-value.
          02 wtime7 pic x(20) value space.
          02 filler pic x value low-value.
      *    
          02 filler pic x(10) value "COLOR8".
          02 filler pic x value low-value.
          02 filler pic x value space.
          02 filler pic x value low-value.
          02 wcolor8 pic x(30) value space.
          02 filler pic x value low-value.
      *    
          02 filler pic x(10) value "NAME8".
          02 filler pic x value low-value.
          02 filler pic x value space.
          02 filler pic x value low-value.
          02 wname8 pic x(30) value space.
          02 filler pic x value low-value.
      *    
          02 filler pic x(10) value "DATA8".
          02 filler pic x value low-value.
          02 filler pic x value space.
          02 filler pic x value low-value.
          02 wdata8 pic x(200) value space.
          02 filler pic x value low-value.
      *    
          02 filler pic x(10) value "TIME8".
          02 filler pic x value low-value.
          02 filler pic x value space.
          02 filler pic x value low-value.
          02 wtime8 pic x(20) value space.
          02 filler pic x value low-value.
      *    
          02 filler pic x(10) value "COLOR9".
          02 filler pic x value low-value.
          02 filler pic x value space.
          02 filler pic x value low-value.
          02 wcolor9 pic x(30) value space.
          02 filler pic x value low-value.
      *    
          02 filler pic x(10) value "NAME9".
          02 filler pic x value low-value.
          02 filler pic x value space.
          02 filler pic x value low-value.
          02 wname9 pic x(30) value space.
          02 filler pic x value low-value.
      *    
          02 filler pic x(10) value "DATA9".
          02 filler pic x value low-value.
          02 filler pic x value space.
          02 filler pic x value low-value.
          02 wdata9 pic x(200) value space.
          02 filler pic x value low-value.
      *    
          02 filler pic x(10) value "TIME9".
          02 filler pic x value low-value.
          02 filler pic x value space.
          02 filler pic x value low-value.
          02 wtime9 pic x(20) value space.
          02 filler pic x value low-value.
      *    
          02 filler pic x(10) value "COLOR10".
          02 filler pic x value low-value.
          02 filler pic x value space.
          02 filler pic x value low-value.
          02 wcolor10 pic x(30) value space.
          02 filler pic x value low-value.
      *    
          02 filler pic x(10) value "NAME10".
          02 filler pic x value low-value.
          02 filler pic x value space.
          02 filler pic x value low-value.
          02 wname10 pic x(30) value space.
          02 filler pic x value low-value.
      *    
          02 filler pic x(10) value "DATA10".
          02 filler pic x value low-value.
          02 filler pic x value space.
          02 filler pic x value low-value.
          02 wdata10 pic x(200) value space.
          02 filler pic x value low-value.
      *    
          02 filler pic x(10) value "TIME10".
          02 filler pic x value low-value.
          02 filler pic x value space.
          02 filler pic x value low-value.
          02 wtime10 pic x(20) value space.
          02 filler pic x value low-value.
      *    
          02 filler pic x(10) value "COLOR11".
          02 filler pic x value low-value.
          02 filler pic x value space.
          02 filler pic x value low-value.
          02 wcolor11 pic x(30) value space.
          02 filler pic x value low-value.
      *    
          02 filler pic x(10) value "NAME11".
          02 filler pic x value low-value.
          02 filler pic x value space.
          02 filler pic x value low-value.
          02 wname11 pic x(30) value space.
          02 filler pic x value low-value.
      *    
          02 filler pic x(10) value "DATA11".
          02 filler pic x value low-value.
          02 filler pic x value space.
          02 filler pic x value low-value.
          02 wdata11 pic x(200) value space.
          02 filler pic x value low-value.
      *    
          02 filler pic x(10) value "TIME11".
          02 filler pic x value low-value.
          02 filler pic x value space.
          02 filler pic x value low-value.
          02 wtime11 pic x(20) value space.
          02 filler pic x value low-value.
      *    
          02 filler pic x(10) value "COLOR12".
          02 filler pic x value low-value.
          02 filler pic x value space.
          02 filler pic x value low-value.
          02 wcolor12 pic x(30) value space.
          02 filler pic x value low-value.
      *    
          02 filler pic x(10) value "NAME12".
          02 filler pic x value low-value.
          02 filler pic x value space.
          02 filler pic x value low-value.
          02 wname12 pic x(30) value space.
          02 filler pic x value low-value.
      *    
          02 filler pic x(10) value "DATA12".
          02 filler pic x value low-value.
          02 filler pic x value space.
          02 filler pic x value low-value.
          02 wdata12 pic x(200) value space.
          02 filler pic x value low-value.
      *    
          02 filler pic x(10) value "TIME12".
          02 filler pic x value low-value.
          02 filler pic x value space.
          02 filler pic x value low-value.
          02 wtime12 pic x(20) value space.
          02 filler pic x value low-value.
      *    
          02 filler pic x(10) value "COLOR13".
          02 filler pic x value low-value.
          02 filler pic x value space.
          02 filler pic x value low-value.
          02 wcolor13 pic x(30) value space.
          02 filler pic x value low-value.
      *    
          02 filler pic x(10) value "NAME13".
          02 filler pic x value low-value.
          02 filler pic x value space.
          02 filler pic x value low-value.
          02 wname13 pic x(30) value space.
          02 filler pic x value low-value.
      *    
          02 filler pic x(10) value "DATA13".
          02 filler pic x value low-value.
          02 filler pic x value space.
          02 filler pic x value low-value.
          02 wdata13 pic x(200) value space.
          02 filler pic x value low-value.
      *    
          02 filler pic x(10) value "TIME13".
          02 filler pic x value low-value.
          02 filler pic x value space.
          02 filler pic x value low-value.
          02 wtime13 pic x(20) value space.
          02 filler pic x value low-value.
      *    
          02 filler pic x(10) value "COLOR14".
          02 filler pic x value low-value.
          02 filler pic x value space.
          02 filler pic x value low-value.
          02 wcolor14 pic x(30) value space.
          02 filler pic x value low-value.
      *    
          02 filler pic x(10) value "NAME14".
          02 filler pic x value low-value.
          02 filler pic x value space.
          02 filler pic x value low-value.
          02 wname14 pic x(30) value space.
          02 filler pic x value low-value.
      *    
          02 filler pic x(10) value "DATA14".
          02 filler pic x value low-value.
          02 filler pic x value space.
          02 filler pic x value low-value.
          02 wdata14 pic x(200) value space.
          02 filler pic x value low-value.
      *    
          02 filler pic x(10) value "TIME14".
          02 filler pic x value low-value.
          02 filler pic x value space.
          02 filler pic x value low-value.
          02 wtime14 pic x(20) value space.
          02 filler pic x value low-value.
      *    
          02 filler pic x(10) value "COLOR15".
          02 filler pic x value low-value.
          02 filler pic x value space.
          02 filler pic x value low-value.
          02 wcolor15 pic x(30) value space.
          02 filler pic x value low-value.
      *    
          02 filler pic x(10) value "NAME15".
          02 filler pic x value low-value.
          02 filler pic x value space.
          02 filler pic x value low-value.
          02 wname15 pic x(30) value space.
          02 filler pic x value low-value.
      *    
          02 filler pic x(10) value "DATA15".
          02 filler pic x value low-value.
          02 filler pic x value space.
          02 filler pic x value low-value.
          02 wdata15 pic x(200) value space.
          02 filler pic x value low-value.
      *    
          02 filler pic x(10) value "TIME15".
          02 filler pic x value low-value.
          02 filler pic x value space.
          02 filler pic x value low-value.
          02 wtime15 pic x(20) value space.
          02 filler pic x value low-value.
      *    
          02 filler pic x(10) value "COLOR16".
          02 filler pic x value low-value.
          02 filler pic x value space.
          02 filler pic x value low-value.
          02 wcolor16 pic x(30) value space.
          02 filler pic x value low-value.
      *    
          02 filler pic x(10) value "NAME16".
          02 filler pic x value low-value.
          02 filler pic x value space.
          02 filler pic x value low-value.
          02 wname16 pic x(30) value space.
          02 filler pic x value low-value.
      *    
          02 filler pic x(10) value "DATA16".
          02 filler pic x value low-value.
          02 filler pic x value space.
          02 filler pic x value low-value.
          02 wdata16 pic x(200) value space.
          02 filler pic x value low-value.
      *    
          02 filler pic x(10) value "TIME16".
          02 filler pic x value low-value.
          02 filler pic x value space.
          02 filler pic x value low-value.
          02 wtime16 pic x(20) value space.
          02 filler pic x value low-value.
      *    
          02 filler pic x(10) value "COLOR17".
          02 filler pic x value low-value.
          02 filler pic x value space.
          02 filler pic x value low-value.
          02 wcolor17 pic x(30) value space.
          02 filler pic x value low-value.
      *    
          02 filler pic x(10) value "NAME17".
          02 filler pic x value low-value.
          02 filler pic x value space.
          02 filler pic x value low-value.
          02 wname17 pic x(30) value space.
          02 filler pic x value low-value.
      *    
          02 filler pic x(10) value "DATA17".
          02 filler pic x value low-value.
          02 filler pic x value space.
          02 filler pic x value low-value.
          02 wdata17 pic x(200) value space.
          02 filler pic x value low-value.
      *    
          02 filler pic x(10) value "TIME17".
          02 filler pic x value low-value.
          02 filler pic x value space.
          02 filler pic x value low-value.
          02 wtime17 pic x(20) value space.
          02 filler pic x value low-value.
      *    
          02 filler pic x(10) value "COLOR18".
          02 filler pic x value low-value.
          02 filler pic x value space.
          02 filler pic x value low-value.
          02 wcolor18 pic x(30) value space.
          02 filler pic x value low-value.
      *    
          02 filler pic x(10) value "NAME18".
          02 filler pic x value low-value.
          02 filler pic x value space.
          02 filler pic x value low-value.
          02 wname18 pic x(30) value space.
          02 filler pic x value low-value.
      *    
          02 filler pic x(10) value "DATA18".
          02 filler pic x value low-value.
          02 filler pic x value space.
          02 filler pic x value low-value.
          02 wdata18 pic x(200) value space.
          02 filler pic x value low-value.
      *    
          02 filler pic x(10) value "TIME18".
          02 filler pic x value low-value.
          02 filler pic x value space.
          02 filler pic x value low-value.
          02 wtime18 pic x(20) value space.
          02 filler pic x value low-value.
      *    
          02 filler pic x(10) value "COLOR19".
          02 filler pic x value low-value.
          02 filler pic x value space.
          02 filler pic x value low-value.
          02 wcolor19 pic x(30) value space.
          02 filler pic x value low-value.
      *    
          02 filler pic x(10) value "NAME19".
          02 filler pic x value low-value.
          02 filler pic x value space.
          02 filler pic x value low-value.
          02 wname19 pic x(30) value space.
          02 filler pic x value low-value.
      *    
          02 filler pic x(10) value "DATA19".
          02 filler pic x value low-value.
          02 filler pic x value space.
          02 filler pic x value low-value.
          02 wdata19 pic x(200) value space.
          02 filler pic x value low-value.
      *    
          02 filler pic x(10) value "TIME19".
          02 filler pic x value low-value.
          02 filler pic x value space.
          02 filler pic x value low-value.
          02 wtime19 pic x(20) value space.
          02 filler pic x value low-value.
      *    
          02 filler pic x(10) value "COLOR20".
          02 filler pic x value low-value.
          02 filler pic x value space.
          02 filler pic x value low-value.
          02 wcolor20 pic x(30) value space.
          02 filler pic x value low-value.
      *    
          02 filler pic x(10) value "NAME20".
          02 filler pic x value low-value.
          02 filler pic x value space.
          02 filler pic x value low-value.
          02 wname20 pic x(30) value space.
          02 filler pic x value low-value.
      *    
          02 filler pic x(10) value "DATA20".
          02 filler pic x value low-value.
          02 filler pic x value space.
          02 filler pic x value low-value.
          02 wdata20 pic x(200) value space.
          02 filler pic x value low-value.
      *    
          02 filler pic x(10) value "TIME20".
          02 filler pic x value low-value.
          02 filler pic x value space.
          02 filler pic x value low-value.
          02 wtime20 pic x(20) value space.
          02 filler pic x value low-value.
      *    
          02 filler pic x(10) value "COLOR21".
          02 filler pic x value low-value.
          02 filler pic x value space.
          02 filler pic x value low-value.
          02 wcolor21 pic x(30) value space.
          02 filler pic x value low-value.
      *    
          02 filler pic x(10) value "NAME21".
          02 filler pic x value low-value.
          02 filler pic x value space.
          02 filler pic x value low-value.
          02 wname21 pic x(30) value space.
          02 filler pic x value low-value.
      *    
          02 filler pic x(10) value "DATA21".
          02 filler pic x value low-value.
          02 filler pic x value space.
          02 filler pic x value low-value.
          02 wdata21 pic x(200) value space.
          02 filler pic x value low-value.
      *    
          02 filler pic x(10) value "TIME21".
          02 filler pic x value low-value.
          02 filler pic x value space.
          02 filler pic x value low-value.
          02 wtime21 pic x(20) value space.
          02 filler pic x value low-value.
      *    
          02 filler pic x(10) value "COLOR22".
          02 filler pic x value low-value.
          02 filler pic x value space.
          02 filler pic x value low-value.
          02 wcolor22 pic x(30) value space.
          02 filler pic x value low-value.
      *    
          02 filler pic x(10) value "NAME22".
          02 filler pic x value low-value.
          02 filler pic x value space.
          02 filler pic x value low-value.
          02 wname22 pic x(30) value space.
          02 filler pic x value low-value.
      *    
          02 filler pic x(10) value "DATA22".
          02 filler pic x value low-value.
          02 filler pic x value space.
          02 filler pic x value low-value.
          02 wdata22 pic x(200) value space.
          02 filler pic x value low-value.
      *    
          02 filler pic x(10) value "TIME22".
          02 filler pic x value low-value.
          02 filler pic x value space.
          02 filler pic x value low-value.
          02 wtime22 pic x(20) value space.
          02 filler pic x value low-value.
      *    
          02 filler pic x(10) value "COLOR23".
          02 filler pic x value low-value.
          02 filler pic x value space.
          02 filler pic x value low-value.
          02 wcolor23 pic x(30) value space.
          02 filler pic x value low-value.
      *    
          02 filler pic x(10) value "NAME23".
          02 filler pic x value low-value.
          02 filler pic x value space.
          02 filler pic x value low-value.
          02 wname23 pic x(30) value space.
          02 filler pic x value low-value.
      *    
          02 filler pic x(10) value "DATA23".
          02 filler pic x value low-value.
          02 filler pic x value space.
          02 filler pic x value low-value.
          02 wdata23 pic x(200) value space.
          02 filler pic x value low-value.
      *    
          02 filler pic x(10) value "TIME23".
          02 filler pic x value low-value.
          02 filler pic x value space.
          02 filler pic x value low-value.
          02 wtime23 pic x(20) value space.
          02 filler pic x value low-value.
      *    
          02 filler pic x(10) value "COLOR24".
          02 filler pic x value low-value.
          02 filler pic x value space.
          02 filler pic x value low-value.
          02 wcolor24 pic x(30) value space.
          02 filler pic x value low-value.
      *    
          02 filler pic x(10) value "NAME24".
          02 filler pic x value low-value.
          02 filler pic x value space.
          02 filler pic x value low-value.
          02 wname24 pic x(30) value space.
          02 filler pic x value low-value.
      *    
          02 filler pic x(10) value "DATA24".
          02 filler pic x value low-value.
          02 filler pic x value space.
          02 filler pic x value low-value.
          02 wdata24 pic x(200) value space.
          02 filler pic x value low-value.
      *    
          02 filler pic x(10) value "TIME24".
          02 filler pic x value low-value.
          02 filler pic x value space.
          02 filler pic x value low-value.
          02 wtime24 pic x(20) value space.
          02 filler pic x value low-value.
      *    
          02 filler pic x(10) value "COLOR25".
          02 filler pic x value low-value.
          02 filler pic x value space.
          02 filler pic x value low-value.
          02 wcolor25 pic x(30) value space.
          02 filler pic x value low-value.
      *    
          02 filler pic x(10) value "NAME25".
          02 filler pic x value low-value.
          02 filler pic x value space.
          02 filler pic x value low-value.
          02 wname25 pic x(30) value space.
          02 filler pic x value low-value.
      *    
          02 filler pic x(10) value "DATA25".
          02 filler pic x value low-value.
          02 filler pic x value space.
          02 filler pic x value low-value.
          02 wdata25 pic x(200) value space.
          02 filler pic x value low-value.
      *    
          02 filler pic x(10) value "TIME25".
          02 filler pic x value low-value.
          02 filler pic x value space.
          02 filler pic x value low-value.
          02 wtime25 pic x(20) value space.
          02 filler pic x value low-value.
      *    
          02 filler pic x(10) value "COLOR26".
          02 filler pic x value low-value.
          02 filler pic x value space.
          02 filler pic x value low-value.
          02 wcolor26 pic x(30) value space.
          02 filler pic x value low-value.
      *    
          02 filler pic x(10) value "NAME26".
          02 filler pic x value low-value.
          02 filler pic x value space.
          02 filler pic x value low-value.
          02 wname26 pic x(30) value space.
          02 filler pic x value low-value.
      *    
          02 filler pic x(10) value "DATA26".
          02 filler pic x value low-value.
          02 filler pic x value space.
          02 filler pic x value low-value.
          02 wdata26 pic x(200) value space.
          02 filler pic x value low-value.
      *    
          02 filler pic x(10) value "TIME26".
          02 filler pic x value low-value.
          02 filler pic x value space.
          02 filler pic x value low-value.
          02 wtime26 pic x(20) value space.
          02 filler pic x value low-value.
      *    
          02 filler pic x(10) value "COLOR27".
          02 filler pic x value low-value.
          02 filler pic x value space.
          02 filler pic x value low-value.
          02 wcolor27 pic x(30) value space.
          02 filler pic x value low-value.
      *    
          02 filler pic x(10) value "NAME27".
          02 filler pic x value low-value.
          02 filler pic x value space.
          02 filler pic x value low-value.
          02 wname27 pic x(30) value space.
          02 filler pic x value low-value.
      *    
          02 filler pic x(10) value "DATA27".
          02 filler pic x value low-value.
          02 filler pic x value space.
          02 filler pic x value low-value.
          02 wdata27 pic x(200) value space.
          02 filler pic x value low-value.
      *    
          02 filler pic x(10) value "TIME27".
          02 filler pic x value low-value.
          02 filler pic x value space.
          02 filler pic x value low-value.
          02 wtime27 pic x(20) value space.
          02 filler pic x value low-value.
      *    
          02 filler pic x(10) value "COLOR28".
          02 filler pic x value low-value.
          02 filler pic x value space.
          02 filler pic x value low-value.
          02 wcolor28 pic x(30) value space.
          02 filler pic x value low-value.
      *    
          02 filler pic x(10) value "NAME28".
          02 filler pic x value low-value.
          02 filler pic x value space.
          02 filler pic x value low-value.
          02 wname28 pic x(30) value space.
          02 filler pic x value low-value.
      *    
          02 filler pic x(10) value "DATA28".
          02 filler pic x value low-value.
          02 filler pic x value space.
          02 filler pic x value low-value.
          02 wdata28 pic x(200) value space.
          02 filler pic x value low-value.
      *    
          02 filler pic x(10) value "TIME28".
          02 filler pic x value low-value.
          02 filler pic x value space.
          02 filler pic x value low-value.
          02 wtime28 pic x(20) value space.
          02 filler pic x value low-value.
      *    
          02 filler pic x(10) value "COLOR29".
          02 filler pic x value low-value.
          02 filler pic x value space.
          02 filler pic x value low-value.
          02 wcolor29 pic x(30) value space.
          02 filler pic x value low-value.
      *    
          02 filler pic x(10) value "NAME29".
          02 filler pic x value low-value.
          02 filler pic x value space.
          02 filler pic x value low-value.
          02 wname29 pic x(30) value space.
          02 filler pic x value low-value.
      *    
          02 filler pic x(10) value "DATA29".
          02 filler pic x value low-value.
          02 filler pic x value space.
          02 filler pic x value low-value.
          02 wdata29 pic x(200) value space.
          02 filler pic x value low-value.
      *    
          02 filler pic x(10) value "TIME29".
          02 filler pic x value low-value.
          02 filler pic x value space.
          02 filler pic x value low-value.
          02 wtime29 pic x(20) value space.
          02 filler pic x value low-value.
      *    
          02 filler pic x(10) value "COLOR30".
          02 filler pic x value low-value.
          02 filler pic x value space.
          02 filler pic x value low-value.
          02 wcolor30 pic x(30) value space.
          02 filler pic x value low-value.
      *    
          02 filler pic x(10) value "NAME30".
          02 filler pic x value low-value.
          02 filler pic x value space.
          02 filler pic x value low-value.
          02 wname30 pic x(30) value space.
          02 filler pic x value low-value.
      *    
          02 filler pic x(10) value "DATA30".
          02 filler pic x value low-value.
          02 filler pic x value space.
          02 filler pic x value low-value.
          02 wdata30 pic x(200) value space.
          02 filler pic x value low-value.
      *    
          02 filler pic x(10) value "TIME30".
          02 filler pic x value low-value.
          02 filler pic x value space.
          02 filler pic x value low-value.
          02 wtime30 pic x(20) value space.
          02 filler pic x value low-value.
      *    
          02 filler pic x(10) value "COLOR31".
          02 filler pic x value low-value.
          02 filler pic x value space.
          02 filler pic x value low-value.
          02 wcolor31 pic x(30) value space.
          02 filler pic x value low-value.
      *    
          02 filler pic x(10) value "NAME31".
          02 filler pic x value low-value.
          02 filler pic x value space.
          02 filler pic x value low-value.
          02 wname31 pic x(30) value space.
          02 filler pic x value low-value.
      *    
          02 filler pic x(10) value "DATA31".
          02 filler pic x value low-value.
          02 filler pic x value space.
          02 filler pic x value low-value.
          02 wdata31 pic x(200) value space.
          02 filler pic x value low-value.
      *    
          02 filler pic x(10) value "TIME31".
          02 filler pic x value low-value.
          02 filler pic x value space.
          02 filler pic x value low-value.
          02 wtime31 pic x(20) value space.
          02 filler pic x value low-value.
      *    
          02 filler pic x(10) value "COLOR32".
          02 filler pic x value low-value.
          02 filler pic x value space.
          02 filler pic x value low-value.
          02 wcolor32 pic x(30) value space.
          02 filler pic x value low-value.
      *    
          02 filler pic x(10) value "NAME32".
          02 filler pic x value low-value.
          02 filler pic x value space.
          02 filler pic x value low-value.
          02 wname32 pic x(30) value space.
          02 filler pic x value low-value.
      *    
          02 filler pic x(10) value "DATA32".
          02 filler pic x value low-value.
          02 filler pic x value space.
          02 filler pic x value low-value.
          02 wdata32 pic x(200) value space.
          02 filler pic x value low-value.
      *    
          02 filler pic x(10) value "TIME32".
          02 filler pic x value low-value.
          02 filler pic x value space.
          02 filler pic x value low-value.
          02 wtime32 pic x(20) value space.
          02 filler pic x value low-value.
      *    
          02 filler pic x(10) value "COLOR33".
          02 filler pic x value low-value.
          02 filler pic x value space.
          02 filler pic x value low-value.
          02 wcolor33 pic x(30) value space.
          02 filler pic x value low-value.
      *    
          02 filler pic x(10) value "NAME33".
          02 filler pic x value low-value.
          02 filler pic x value space.
          02 filler pic x value low-value.
          02 wname33 pic x(30) value space.
          02 filler pic x value low-value.
      *    
          02 filler pic x(10) value "DATA33".
          02 filler pic x value low-value.
          02 filler pic x value space.
          02 filler pic x value low-value.
          02 wdata33 pic x(200) value space.
          02 filler pic x value low-value.
      *    
          02 filler pic x(10) value "TIME33".
          02 filler pic x value low-value.
          02 filler pic x value space.
          02 filler pic x value low-value.
          02 wtime33 pic x(20) value space.
          02 filler pic x value low-value.
      *    
          02 filler pic x(10) value "COLOR34".
          02 filler pic x value low-value.
          02 filler pic x value space.
          02 filler pic x value low-value.
          02 wcolor34 pic x(30) value space.
          02 filler pic x value low-value.
      *    
          02 filler pic x(10) value "NAME34".
          02 filler pic x value low-value.
          02 filler pic x value space.
          02 filler pic x value low-value.
          02 wname34 pic x(30) value space.
          02 filler pic x value low-value.
      *    
          02 filler pic x(10) value "DATA34".
          02 filler pic x value low-value.
          02 filler pic x value space.
          02 filler pic x value low-value.
          02 wdata34 pic x(200) value space.
          02 filler pic x value low-value.
      *    
          02 filler pic x(10) value "TIME34".
          02 filler pic x value low-value.
          02 filler pic x value space.
          02 filler pic x value low-value.
          02 wtime34 pic x(20) value space.
          02 filler pic x value low-value.
      *    
          02 filler pic x(10) value "COLOR35".
          02 filler pic x value low-value.
          02 filler pic x value space.
          02 filler pic x value low-value.
          02 wcolor35 pic x(30) value space.
          02 filler pic x value low-value.
      *    
          02 filler pic x(10) value "NAME35".
          02 filler pic x value low-value.
          02 filler pic x value space.
          02 filler pic x value low-value.
          02 wname35 pic x(30) value space.
          02 filler pic x value low-value.
      *    
          02 filler pic x(10) value "DATA35".
          02 filler pic x value low-value.
          02 filler pic x value space.
          02 filler pic x value low-value.
          02 wdata35 pic x(200) value space.
          02 filler pic x value low-value.
      *    
          02 filler pic x(10) value "TIME35".
          02 filler pic x value low-value.
          02 filler pic x value space.
          02 filler pic x value low-value.
          02 wtime35 pic x(20) value space.
          02 filler pic x value low-value.
      *    
          02 filler pic x(10) value "COLOR36".
          02 filler pic x value low-value.
          02 filler pic x value space.
          02 filler pic x value low-value.
          02 wcolor36 pic x(30) value space.
          02 filler pic x value low-value.
      *    
          02 filler pic x(10) value "NAME36".
          02 filler pic x value low-value.
          02 filler pic x value space.
          02 filler pic x value low-value.
          02 wname36 pic x(30) value space.
          02 filler pic x value low-value.
      *    
          02 filler pic x(10) value "DATA36".
          02 filler pic x value low-value.
          02 filler pic x value space.
          02 filler pic x value low-value.
          02 wdata36 pic x(200) value space.
          02 filler pic x value low-value.
      *    
          02 filler pic x(10) value "TIME36".
          02 filler pic x value low-value.
          02 filler pic x value space.
          02 filler pic x value low-value.
          02 wtime36 pic x(20) value space.
          02 filler pic x value low-value.
      *    
          02 filler pic x(10) value "COLOR37".
          02 filler pic x value low-value.
          02 filler pic x value space.
          02 filler pic x value low-value.
          02 wcolor37 pic x(30) value space.
          02 filler pic x value low-value.
      *    
          02 filler pic x(10) value "NAME37".
          02 filler pic x value low-value.
          02 filler pic x value space.
          02 filler pic x value low-value.
          02 wname37 pic x(30) value space.
          02 filler pic x value low-value.
      *    
          02 filler pic x(10) value "DATA37".
          02 filler pic x value low-value.
          02 filler pic x value space.
          02 filler pic x value low-value.
          02 wdata37 pic x(200) value space.
          02 filler pic x value low-value.
      *    
          02 filler pic x(10) value "TIME37".
          02 filler pic x value low-value.
          02 filler pic x value space.
          02 filler pic x value low-value.
          02 wtime37 pic x(20) value space.
          02 filler pic x value low-value.
      *    
          02 filler pic x(10) value "COLOR38".
          02 filler pic x value low-value.
          02 filler pic x value space.
          02 filler pic x value low-value.
          02 wcolor38 pic x(30) value space.
          02 filler pic x value low-value.
      *    
          02 filler pic x(10) value "NAME38".
          02 filler pic x value low-value.
          02 filler pic x value space.
          02 filler pic x value low-value.
          02 wname38 pic x(30) value space.
          02 filler pic x value low-value.
      *    
          02 filler pic x(10) value "DATA38".
          02 filler pic x value low-value.
          02 filler pic x value space.
          02 filler pic x value low-value.
          02 wdata38 pic x(200) value space.
          02 filler pic x value low-value.
      *    
          02 filler pic x(10) value "TIME38".
          02 filler pic x value low-value.
          02 filler pic x value space.
          02 filler pic x value low-value.
          02 wtime38 pic x(20) value space.
          02 filler pic x value low-value.
      *    
          02 filler pic x(10) value "COLOR39".
          02 filler pic x value low-value.
          02 filler pic x value space.
          02 filler pic x value low-value.
          02 wcolor39 pic x(30) value space.
          02 filler pic x value low-value.
      *    
          02 filler pic x(10) value "NAME39".
          02 filler pic x value low-value.
          02 filler pic x value space.
          02 filler pic x value low-value.
          02 wname39 pic x(30) value space.
          02 filler pic x value low-value.
      *    
          02 filler pic x(10) value "DATA39".
          02 filler pic x value low-value.
          02 filler pic x value space.
          02 filler pic x value low-value.
          02 wdata39 pic x(200) value space.
          02 filler pic x value low-value.
      *    
          02 filler pic x(10) value "TIME39".
          02 filler pic x value low-value.
          02 filler pic x value space.
          02 filler pic x value low-value.
          02 wtime39 pic x(20) value space.
          02 filler pic x value low-value.
      *    
          02 filler pic x(10) value "COLOR40".
          02 filler pic x value low-value.
          02 filler pic x value space.
          02 filler pic x value low-value.
          02 wcolor40 pic x(30) value space.
          02 filler pic x value low-value.
      *    
          02 filler pic x(10) value "NAME40".
          02 filler pic x value low-value.
          02 filler pic x value space.
          02 filler pic x value low-value.
          02 wname40 pic x(30) value space.
          02 filler pic x value low-value.
      *    
          02 filler pic x(10) value "DATA40".
          02 filler pic x value low-value.
          02 filler pic x value space.
          02 filler pic x value low-value.
          02 wdata40 pic x(200) value space.
          02 filler pic x value low-value.
      *    
          02 filler pic x(10) value "TIME40".
          02 filler pic x value low-value.
          02 filler pic x value space.
          02 filler pic x value low-value.
          02 wtime40 pic x(20) value space.
          02 filler pic x value low-value.
      *
          02 filler pic x(20) value 'SCREENNAME'.
          02 filler pic x value low-value.
          02 filler pic x value space.
          02 filler pic x value low-value.
          02 wscreenname pic x(200) value space.
          02 filler pic x value low-value.
      *
          02 filler pic x(10) value "MODE".
          02 filler pic x value low-value.
          02 filler pic x value space.
          02 filler pic x value low-value.
          02 wmode pic x(10) value space.
          02 filler pic x value low-value.
      *
          02 filler pic x value low-value.

      01 chat-param-in.
          02 wname pic x(30) value space.
          02 wsenddata pic x(200) value space.
          02 wcolor  pic x(30) value space.
          02 wcolor1 pic x(30) value space.
          02 wname1 pic x(30) value space.
          02 wdata1 pic x(200) value space.
          02 wtime1 pic x(20) value space.
          02 wcolor2 pic x(30) value space.
          02 wname2 pic x(30) value space.
          02 wdata2 pic x(200) value space.
          02 wtime2 pic x(20) value space.
          02 wcolor3 pic x(30) value space.
          02 wname3 pic x(30) value space.
          02 wdata3 pic x(200) value space.
          02 wtime3 pic x(20) value space.
          02 wcolor4 pic x(30) value space.
          02 wname4 pic x(30) value space.
          02 wdata4 pic x(200) value space.
          02 wtime4 pic x(20) value space.
          02 wcolor5 pic x(30) value space.
          02 wname5 pic x(30) value space.
          02 wdata5 pic x(200) value space.
          02 wtime5 pic x(20) value space.
          02 wcolor6 pic x(30) value space.
          02 wname6 pic x(30) value space.
          02 wdata6 pic x(200) value space.
          02 wtime6 pic x(20) value space.
          02 wcolor7 pic x(30) value space.
          02 wname7 pic x(30) value space.
          02 wdata7 pic x(200) value space.
          02 wtime7 pic x(20) value space.
          02 wcolor8 pic x(30) value space.
          02 wname8 pic x(30) value space.
          02 wdata8 pic x(200) value space.
          02 wtime8 pic x(20) value space.
          02 wcolor9 pic x(30) value space.
          02 wname9 pic x(30) value space.
          02 wdata9 pic x(200) value space.
          02 wtime9 pic x(20) value space.
          02 wcolor10 pic x(30) value space.
          02 wname10 pic x(30) value space.
          02 wdata10 pic x(200) value space.
          02 wtime10 pic x(20) value space.
          02 wcolor11 pic x(30) value space.
          02 wname11 pic x(30) value space.
          02 wdata11 pic x(200) value space.
          02 wtime11 pic x(20) value space.
          02 wcolor12 pic x(30) value space.
          02 wname12 pic x(30) value space.
          02 wdata12 pic x(200) value space.
          02 wtime12 pic x(20) value space.
          02 wcolor13 pic x(30) value space.
          02 wname13 pic x(30) value space.
          02 wdata13 pic x(200) value space.
          02 wtime13 pic x(20) value space.
          02 wcolor14 pic x(30) value space.
          02 wname14 pic x(30) value space.
          02 wdata14 pic x(200) value space.
          02 wtime14 pic x(20) value space.
          02 wcolor15 pic x(30) value space.
          02 wname15 pic x(30) value space.
          02 wdata15 pic x(200) value space.
          02 wtime15 pic x(20) value space.
          02 wcolor16 pic x(30) value space.
          02 wname16 pic x(30) value space.
          02 wdata16 pic x(200) value space.
          02 wtime16 pic x(20) value space.
          02 wcolor17 pic x(30) value space.
          02 wname17 pic x(30) value space.
          02 wdata17 pic x(200) value space.
          02 wtime17 pic x(20) value space.
          02 wcolor18 pic x(30) value space.
          02 wname18 pic x(30) value space.
          02 wdata18 pic x(200) value space.
          02 wtime18 pic x(20) value space.
          02 wcolor19 pic x(30) value space.
          02 wname19 pic x(30) value space.
          02 wdata19 pic x(200) value space.
          02 wtime19 pic x(20) value space.
          02 wcolor20 pic x(30) value space.
          02 wname20 pic x(30) value space.
          02 wdata20 pic x(200) value space.
          02 wtime20 pic x(20) value space.
          02 wcolor21 pic x(30) value space.
          02 wname21 pic x(30) value space.
          02 wdata21 pic x(200) value space.
          02 wtime21 pic x(20) value space.
          02 wcolor22 pic x(30) value space.
          02 wname22 pic x(30) value space.
          02 wdata22 pic x(200) value space.
          02 wtime22 pic x(20) value space.
          02 wcolor23 pic x(30) value space.
          02 wname23 pic x(30) value space.
          02 wdata23 pic x(200) value space.
          02 wtime23 pic x(20) value space.
          02 wcolor24 pic x(30) value space.
          02 wname24 pic x(30) value space.
          02 wdata24 pic x(200) value space.
          02 wtime24 pic x(20) value space.
          02 wcolor25 pic x(30) value space.
          02 wname25 pic x(30) value space.
          02 wdata25 pic x(200) value space.
          02 wtime25 pic x(20) value space.
          02 wcolor26 pic x(30) value space.
          02 wname26 pic x(30) value space.
          02 wdata26 pic x(200) value space.
          02 wtime26 pic x(20) value space.
          02 wcolor27 pic x(30) value space.
          02 wname27 pic x(30) value space.
          02 wdata27 pic x(200) value space.
          02 wtime27 pic x(20) value space.
          02 wcolor28 pic x(30) value space.
          02 wname28 pic x(30) value space.
          02 wdata28 pic x(200) value space.
          02 wtime28 pic x(20) value space.
          02 wcolor29 pic x(30) value space.
          02 wname29 pic x(30) value space.
          02 wdata29 pic x(200) value space.
          02 wtime29 pic x(20) value space.
          02 wcolor30 pic x(30) value space.
          02 wname30 pic x(30) value space.
          02 wdata30 pic x(200) value space.
          02 wtime30 pic x(20) value space.
          02 wcolor31 pic x(30) value space.
          02 wname31 pic x(30) value space.
          02 wdata31 pic x(200) value space.
          02 wtime31 pic x(20) value space.
          02 wcolor32 pic x(30) value space.
          02 wname32 pic x(30) value space.
          02 wdata32 pic x(200) value space.
          02 wtime32 pic x(20) value space.
          02 wcolor33 pic x(30) value space.
          02 wname33 pic x(30) value space.
          02 wdata33 pic x(200) value space.
          02 wtime33 pic x(20) value space.
          02 wcolor34 pic x(30) value space.
          02 wname34 pic x(30) value space.
          02 wdata34 pic x(200) value space.
          02 wtime34 pic x(20) value space.
          02 wcolor35 pic x(30) value space.
          02 wname35 pic x(30) value space.
          02 wdata35 pic x(200) value space.
          02 wtime35 pic x(20) value space.
          02 wcolor36 pic x(30) value space.
          02 wname36 pic x(30) value space.
          02 wdata36 pic x(200) value space.
          02 wtime36 pic x(20) value space.
          02 wcolor37 pic x(30) value space.
          02 wname37 pic x(30) value space.
          02 wdata37 pic x(200) value space.
          02 wtime37 pic x(20) value space.
          02 wcolor38 pic x(30) value space.
          02 wname38 pic x(30) value space.
          02 wdata38 pic x(200) value space.
          02 wtime38 pic x(20) value space.
          02 wcolor39 pic x(30) value space.
          02 wname39 pic x(30) value space.
          02 wdata39 pic x(200) value space.
          02 wtime39 pic x(20) value space.
          02 wcolor40 pic x(30) value space.
          02 wname40 pic x(30) value space.
          02 wdata40 pic x(200) value space.
          02 wtime40 pic x(20) value space.
          02 wscreenname pic x(200) value space.
          02 wmode pic x(10) value space.
      *
       01 chat-param-in-tab redefines chat-param-in.
          02 wname pic x(30).
          02 wsenddata pic x(200).
          02 wcolor pic x(30).
          02 dat-table.
            03 dat-table-item  occurs 40.
              04 wcolor pic x(30).
              04 wname pic x(30).
              04 wdata pic x(200).
              04 wtime pic x(20).
          02 wscreenname pic x(200).
          02 wmode2       pic x(10).
      *
       01 chat-param-ot.
          02 wrefresh pic x(30) value space.
          02 wname pic x(30) value space.
          02 wsenddata pic x(200) value space.
          02 wcolor1 pic x(30) value space.
          02 wname1 pic x(30) value space.
          02 wdata1 pic x(200) value space.
          02 wtime1 pic x(20) value space.
          02 wcolor2 pic x(30) value space.
          02 wname2 pic x(30) value space.
          02 wdata2 pic x(200) value space.
          02 wtime2 pic x(20) value space.
          02 wcolor3 pic x(30) value space.
          02 wname3 pic x(30) value space.
          02 wdata3 pic x(200) value space.
          02 wtime3 pic x(20) value space.
          02 wcolor4 pic x(30) value space.
          02 wname4 pic x(30) value space.
          02 wdata4 pic x(200) value space.
          02 wtime4 pic x(20) value space.
          02 wcolor5 pic x(30) value space.
          02 wname5 pic x(30) value space.
          02 wdata5 pic x(200) value space.
          02 wtime5 pic x(20) value space.
          02 wcolor6 pic x(30) value space.
          02 wname6 pic x(30) value space.
          02 wdata6 pic x(200) value space.
          02 wtime6 pic x(20) value space.
          02 wcolor7 pic x(30) value space.
          02 wname7 pic x(30) value space.
          02 wdata7 pic x(200) value space.
          02 wtime7 pic x(20) value space.
          02 wcolor8 pic x(30) value space.
          02 wname8 pic x(30) value space.
          02 wdata8 pic x(200) value space.
          02 wtime8 pic x(20) value space.
          02 wcolor9 pic x(30) value space.
          02 wname9 pic x(30) value space.
          02 wdata9 pic x(200) value space.
          02 wtime9 pic x(20) value space.
          02 wcolor10 pic x(30) value space.
          02 wname10 pic x(30) value space.
          02 wdata10 pic x(200) value space.
          02 wtime10 pic x(20) value space.
          02 wcolor11 pic x(30) value space.
          02 wname11 pic x(30) value space.
          02 wdata11 pic x(200) value space.
          02 wtime11 pic x(20) value space.
          02 wcolor12 pic x(30) value space.
          02 wname12 pic x(30) value space.
          02 wdata12 pic x(200) value space.
          02 wtime12 pic x(20) value space.
          02 wcolor13 pic x(30) value space.
          02 wname13 pic x(30) value space.
          02 wdata13 pic x(200) value space.
          02 wtime13 pic x(20) value space.
          02 wcolor14 pic x(30) value space.
          02 wname14 pic x(30) value space.
          02 wdata14 pic x(200) value space.
          02 wtime14 pic x(20) value space.
          02 wcolor15 pic x(30) value space.
          02 wname15 pic x(30) value space.
          02 wdata15 pic x(200) value space.
          02 wtime15 pic x(20) value space.
          02 wcolor16 pic x(30) value space.
          02 wname16 pic x(30) value space.
          02 wdata16 pic x(200) value space.
          02 wtime16 pic x(20) value space.
          02 wcolor17 pic x(30) value space.
          02 wname17 pic x(30) value space.
          02 wdata17 pic x(200) value space.
          02 wtime17 pic x(20) value space.
          02 wcolor18 pic x(30) value space.
          02 wname18 pic x(30) value space.
          02 wdata18 pic x(200) value space.
          02 wtime18 pic x(20) value space.
          02 wcolor19 pic x(30) value space.
          02 wname19 pic x(30) value space.
          02 wdata19 pic x(200) value space.
          02 wtime19 pic x(20) value space.
          02 wcolor20 pic x(30) value space.
          02 wname20 pic x(30) value space.
          02 wdata20 pic x(200) value space.
          02 wtime20 pic x(20) value space.
          02 wcolor21 pic x(30) value space.
          02 wname21 pic x(30) value space.
          02 wdata21 pic x(200) value space.
          02 wtime21 pic x(20) value space.
          02 wcolor22 pic x(30) value space.
          02 wname22 pic x(30) value space.
          02 wdata22 pic x(200) value space.
          02 wtime22 pic x(20) value space.
          02 wcolor23 pic x(30) value space.
          02 wname23 pic x(30) value space.
          02 wdata23 pic x(200) value space.
          02 wtime23 pic x(20) value space.
          02 wcolor24 pic x(30) value space.
          02 wname24 pic x(30) value space.
          02 wdata24 pic x(200) value space.
          02 wtime24 pic x(20) value space.
          02 wcolor25 pic x(30) value space.
          02 wname25 pic x(30) value space.
          02 wdata25 pic x(200) value space.
          02 wtime25 pic x(20) value space.
          02 wcolor26 pic x(30) value space.
          02 wname26 pic x(30) value space.
          02 wdata26 pic x(200) value space.
          02 wtime26 pic x(20) value space.
          02 wcolor27 pic x(30) value space.
          02 wname27 pic x(30) value space.
          02 wdata27 pic x(200) value space.
          02 wtime27 pic x(20) value space.
          02 wcolor28 pic x(30) value space.
          02 wname28 pic x(30) value space.
          02 wdata28 pic x(200) value space.
          02 wtime28 pic x(20) value space.
          02 wcolor29 pic x(30) value space.
          02 wname29 pic x(30) value space.
          02 wdata29 pic x(200) value space.
          02 wtime29 pic x(20) value space.
          02 wcolor30 pic x(30) value space.
          02 wname30 pic x(30) value space.
          02 wdata30 pic x(200) value space.
          02 wtime30 pic x(20) value space.
          02 wcolor31 pic x(30) value space.
          02 wname31 pic x(30) value space.
          02 wdata31 pic x(200) value space.
          02 wtime31 pic x(20) value space.
          02 wcolor32 pic x(30) value space.
          02 wname32 pic x(30) value space.
          02 wdata32 pic x(200) value space.
          02 wtime32 pic x(20) value space.
          02 wcolor33 pic x(30) value space.
          02 wname33 pic x(30) value space.
          02 wdata33 pic x(200) value space.
          02 wtime33 pic x(20) value space.
          02 wcolor34 pic x(30) value space.
          02 wname34 pic x(30) value space.
          02 wdata34 pic x(200) value space.
          02 wtime34 pic x(20) value space.
          02 wcolor35 pic x(30) value space.
          02 wname35 pic x(30) value space.
          02 wdata35 pic x(200) value space.
          02 wtime35 pic x(20) value space.
          02 wcolor36 pic x(30) value space.
          02 wname36 pic x(30) value space.
          02 wdata36 pic x(200) value space.
          02 wtime36 pic x(20) value space.
          02 wcolor37 pic x(30) value space.
          02 wname37 pic x(30) value space.
          02 wdata37 pic x(200) value space.
          02 wtime37 pic x(20) value space.
          02 wcolor38 pic x(30) value space.
          02 wname38 pic x(30) value space.
          02 wdata38 pic x(200) value space.
          02 wtime38 pic x(20) value space.
          02 wcolor39 pic x(30) value space.
          02 wname39 pic x(30) value space.
          02 wdata39 pic x(200) value space.
          02 wtime39 pic x(20) value space.
          02 wcolor40 pic x(30) value space.
          02 wname40 pic x(30) value space.
          02 wdata40 pic x(200) value space.
          02 wtime40 pic x(20) value space.
          02 wscreenname pic x(200) value space.
          02 wmode pic x(10) value space.
\end{verbatim}



\end{document}







