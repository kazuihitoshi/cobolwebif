
\begin{itemize}
\item{書式}

{\gt COOKIEコピー句の読込}

\begin{tabular}{|l|}
\hline
WORKING-STORAGE SECTION.\\
COPY web-cookie.\\
77  rc  pic s9(10).\\
\hline
\end{tabular}

{\gt クッキーのセッティング}

\begin{tabular}{|l|}
\hline
INITIALIZE WEB-COOKIE.
MOVE 'クッキー変数名' TO WEB-NAME.\\
MOVE '変数内容'       TO WEB-VALUE.\\
MOVE '変数の期限'     TO WEB-EXPIRES.\\
MOVE '有効パス'       TO WEB-PATH.\\
MOVE 'ドメイン名'     TO WEB-DOMAIN.\\
MOVE 'SECURE'         TO WEB-SECURE.\\
CALL 'WEB\_SET\_COOKIE' using WEB-COOKIE rc.\\
\hline
\end{tabular}

{\gt 受信クッキーの取得}

\begin{tabular}{|l|}
\hline
CALL 'WEB\_POP\_COOKIE' using WEB\_POP,WEB\_PUSH変数領域名.\\
\hline
\end{tabular}

\item{機能}

クッキーの設定及び取得機能を実現する。
クッキーはブラウザに記憶させる変数であり、クッキー設定を行ったサイトを開く際にブラウザからサーバに送信される物です。

クッキーには有効期限をつけたり、サイトのディレクトリパスを限定する事も可能です。

また、セキュアサイトでのみ有効とする設定も可能です。

クッキーの実際の受信はCALL 'WEB\_GET\_QUERY\_STRING'にて行います。
その後、CALL 'WEB\_POP\_COOKIE'にて変数の値を取り出します。

クッキーの設定ははCALL 'WEB\_SET\_COOKIE'にて行いますが、実際にブラウザに値が送信されるのは

CALL 'WEB\_SHOW'が実行された時と、CALL 'WEB\_SEND\_COOKIE'が実行されたときです。

CALL 'WEB\_SHOW'はSCREENNAMEで指定されるhtmlファイルをWEB\_PUSH,WEB\_POP領域と合成してブラウザへ送信します。

DISPLAYステートメント等によりブラウザ送信用のHTMLファイルを生成する場合はWEB\_SEND\_COOOKIEを使用してください。

\vspace{1em}


\end{itemize}



