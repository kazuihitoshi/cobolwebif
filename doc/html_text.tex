\begin{itemize}

\item{表記方法}

\verb+<input type="text" name="CGI変数名" maxlength="データ長" size="表示サイズ" value="@変数名称">+

\item{説明}

入力フィールドの定義します。

\item{コーディング例}


{\gt html}

\begin{tabular}{|l|}
\hline
\verb+<html lang="ja" dir="ltr">+\\
\verb+<meta http-equiv="content-type" content="text/html;charset=x-euc-jp">+\\
\verb+<form action="test">+\\
\verb+<input type="text" name="testdata" maxlength="80" size="80" value="@testdata">+\\
\verb+<input type="submit" value="実行">+\\
\verb+</form>+\\
\verb+</html>+\\
\hline
\end{tabular}

{\gt cobol} source test.cob

\begin{tabular}{|l|}
\hline
\verb+ working-storage section.+\\
\verb+  01 web-if-rec.+\\
\verb+   02 filler pic x(20) value 'testdata'.+\\
\verb+   02 filler pic x value low-value.+\\
\verb+   02 filler pic x value space.+\\
\verb+   02 filler pic x value low-value+\\.
\verb+   02 testdata pic x(80) value space.+\\
\verb+   02 filler pic x value low-value.+\\
\verb+   02 filler pic x value low-value.+\\
\verb++\\
\verb+ procedure division.+\\
\verb+ main.+\\
\verb+  call 'WEB_QUERY_STRING'.+\\
\verb+  call 'WEB_POP' using web-if-rec.+\\
\verb+* 処理+\\
\verb+  call 'WEB_PUSH' using web-if-rec.+\\
\verb+  call 'WEB_SHOW'.+\\
\verb+  stop.+\\
\hline
\end{tabular}

\end{itemize}

