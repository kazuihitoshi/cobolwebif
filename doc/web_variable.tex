
Web APIグローバル領域とCOBOL変数領域の引き渡し空間を定義します。

変数領域は{\it WORKING-STRAGE SECTION.}に定義します。

定義方式は

変数名称1 + NULL + 変数属性1 + NULL + 変数領域1 + NULL + 変数名称2 + NULL + 変数属性2 + NULL + 変数領域2 + NULL .... 変数名称n + NULL + 変数属性n + NULL + 変数領域n + NULL + NULL

のメモリイメージになるように変数領域を記述する。最後のNULLは2つ連続していなければなりません。


変数属性は今後の拡張領域のため、現時点ではスペースを記述しておけば問題ありません。

将来的には左寄せ(デフォルトは右寄せ)や、数値のみ入力などの制御が可能となる予定です。

コーディング書式は次のようになります。

{\gt コーディング書式}

{\footnotesize
\begin{tabular}{|l|}
\hline
 working-strage section.                                              \\
 01 web-if-rec.                                                       \\
   02  filler pic x(変数名称のデータサイズ) value '変数名称'.           \\
   02  filler pic x     value low-value.                              \\
   02  filler pic x     value space.                                  \\
   02  filler pic x     value low-value.                              \\
   02 変数名のデータ領域名 pic x(変数名データサイズ) value space.       \\
   02  filler pic x     value low-value.                              \\
   02  filler pic x(変数名称2データサイズ) value '変数名称2'.       \\
   02  filler pic x     value low-value.                              \\
   02  filler pic x     value space.                                  \\
   02  filler pic x     value low-value.                              \\
   02 変数名2のデータ領域名  pic x(変数名2データサイズ) value space.      \\
   02  filler pic x     value low-value.                              \\
            :       :                                                 \\
   02  filler pic x(変数名nの名称データサイズ) value '変数名n'.       \\
   02  filler pic x     value low-value.                              \\
   02  filler pic x     value space.                                  \\
   02  filler pic x     value low-value.                              \\
   02 変数名nのデータ領域名  pic x(変数名nデータサイズ) value space.      \\
   02  filler pic x     value low-value.                              \\
   02 filler pic x     value low-value.                               \\
\hline
\end{tabular}
}

\vspace{0.5em}

{\small
追記事項:

\begin{itemize}
\item{変数名称 ・・・半角英数字のみで表記、変数名の英字は大文字小文字の区別は無い。}
\item{変数名のデータ領域名 ・・・実データが保存される領域。}
\end{itemize}
}

{\gt 例 CGIよりNAME,EMAIL変数が渡される場合のコーディング例}

\vspace{0.5em}

{\footnotesize
\begin{tabular}{|l|r|l|}
\hline
入力項目名 & 桁数 & 変数名 \\\hline
ATAI1       &  12  & ATAI1  \\\hline
ATAI2       &  12  & ATAI2  \\\hline
KOTAE       &  13  & KOTAE  \\
\hline
\end{tabular}
}
\vspace{1em}

コーディング例は以下の様になります。

\vspace{0.5em}

{\footnotesize
\begin{tabular}{|l|}
\hline
\verb+working-strage section.+\\
\verb+01 web-if-rec.+\\
\verb+ 02 filler pic x(5)  value 'ATAI1'.+\\
\verb+ 02 filler pic x     value low-value.+\\
\verb+ 02 filler pic x     value space.+\\
\verb+ 02 filler pic x     value low-value.+\\
\verb+ 02 ATAI1  pic x(12) value space.+\\
\verb+ 02 filler pic x     value low-value.+\\
\verb+ 02 filler pic x(5)  value 'ATAI2'.+\\
\verb+ 02 filler pic x     value low-value.+\\
\verb+ 02 filler pic x     value space.+\\
\verb+ 02 filler pic x     value low-value.+\\
\verb+ 02 ATAI2  pic x(12) value space.+\\
\verb+ 02 filler pic x     value low-value.+\\
\verb+ 02 filler pic x(5)  value 'KOTAE'.+\\
\verb+ 02 filler pic x     value low-value.+\\
\verb+ 02 filler pic x     value space.+\\
\verb+ 02 filler pic x     value low-value.+\\
\verb+ 02 KOTAE  pic x(13) value space.+\\
\verb+ 02 filler pic x     value low-value.+\\
\verb+ 02 filler pic x     value low-value.+\\
\hline
\end{tabular}
}

このWEB\_POP,WEB\_PUSH変数領域は、実際には以下の様にして使用します。
\vspace{1em}

{\footnotesize
\begin{tabular}{|l|}
\hline
\verb+working-strage section.+\\
\verb+01 web-if-rec.+\\
\verb+ 02 filler pic x(5)  value 'ATAI1'.+\\
\verb+ 02 filler pic x     value low-value.+\\
\verb+ 02 filler pic x     value space.+\\
\verb+ 02 filler pic x     value low-value.+\\
\verb+ 02 ATAI1  pic x(12) value space.+\\
\verb+ 02 filler pic x     value low-value.+\\
\verb+ 02 filler pic x(5)  value 'ATAI2'.+\\
\verb+ 02 filler pic x     value low-value.+\\
\verb+ 02 filler pic x     value space.+\\
\verb+ 02 filler pic x     value low-value.+\\
\verb+ 02 ATAI2  pic x(12) value space.+\\
\verb+ 02 filler pic x     value low-value.+\\
\verb+ 02 filler pic x(5)  value 'KOTAE'.+\\
\verb+ 02 filler pic x     value low-value.+\\
\verb+ 02 filler pic x     value space.+\\
\verb+ 02 filler pic x     value low-value.+\\
\verb+ 02 KOTAE  pic x(13) value space.+\\
\verb+ 02 filler pic x     value low-value.+\\
\verb+ 02 filler pic x     value low-value.+\\
\verb+01 web-in-rec.+\\
\verb+  02 ATAI1   pic 9(12).+\\
\verb+  02 ATAI2   pic 9(12).+\\
\verb+  02 KOTAE   pic 9(12).+\\
\verb+01 web-ot-rec.+\\
\verb+  02 ATAI1   pic ZZZZZZZZZZZ9.+\\
\verb+  02 ATAI2   pic ZZZZZZZZZZZ9.+\\
\verb+  02 KOTAE   pic ZZZZZZZZZZZ9.+\\
\verb+ procedure division.+\\
\verb+    call 'WEB_GET_QUERY_STRING'+\\
\verb+    call 'WEB_POP' using web-if-rec.+\\
\verb+    move corr web-if-rec to web-in-rec.+\\
  処理 サンプルなので足し算を行う\\
\verb$    compute KOTAE of web-in-rec = ATAI1 of web-in-rec + ATAI2 of web-in-rec.$\\
  出力用に数値を編集\\
\verb+   move corr web-in-rec to web-ot-rec.+\\
WEBインタフェース領域へコピー\\
\verb+   move corr web-ot-rec to web-if-rec.+\\
\verb+   call 'WEB_PUSH' using web-if-rec.+\\
\verb+   call 'WEB_SHOW'.+\\
\verb+   stop run.+\\
\hline
\end{tabular}
}

流れを示すと、

{\small

\begin{itemize}

\item[(1)]{CGI変数を受け取る。}
\item[(2)]{web-if-rec エリアにCGIから受け取った値を格納する。}
\item[(3)]{web-if-rec エリアを処理で扱いやすい web-in-rec へコピーする。}
\item[(4)]{web-in-recをベースに処理を行う。}
\item[(5)]{web-in-recの内容を出力形式編集領域web-ot-recへコピーする。}
\item[(6)]{web-ot-recをCGIエリアweb-in-recへ転送する。(web-in-recには所定の様式編集がなされた値が入る)}
\item[(7)]{web-in-recを元にHTMLを合成する。}
\end{itemize}
}

となります。
