\subsubsection{Webサーバ AN HTTPDのインストール}

AN HTTPD\footnote{AN HTTPDは中田昭雄氏作成のWebサーバソフトです。}はWindowsで利用できるWebサーバの一つです。

ダウンロードの方法

http://www.st.rim.or.jp/~nakata/よりダウンロード出来ます。
執筆時の最新バージョンはVersion1.42gで、ファイル名はhttpd142g.exeです。

このファイルをダウンロードし、ダブルクリックにて起動してください。

以上でインストールは完了です。

\subsubsection{MinGW,OpenCOBOLのインストール}

MinGWとはMinimalist GNU for Windowsの略であり、GCCによるネイティブな32bit Windowsプログラムを生成するための開発環境です。

MinGWと似たものにCygwinという物がありますが、MinGWで生成したEXEはMinGW用のDLLは不要です。
EXEのみで動作可能なネイティブコードを生成することが可能です。

このあたりの詳細は ビットウォークさんのホームページ

{\it http://members10.tsukaeru.net/bitwalk/index.html →MinGW→MinGWについて}

に詳細の記述があります。

詳細はこちらを参照してみてください。

\vspace{1em}

\begin{itemize}

\item{MinGW,tools,open-cobolをダウンロードする}
\footnote{MinGWはhttp://member10.tsukaeru.net/bitwalk/index.htmlよりCD-Rにて購入する事も可能です。}

http://members10.tsukaeru.net/bitwalk/index.html → 「MinGw」をクリック → 「ダウンロード」をクリック
→ 「mingw\_017.exe」「mingw\_tools\_004.exe」「open-cobol-0.10.1\_011.exe」をそれぞれダウロードします。

\item{mingwのインストール}

ダウンロードした{\it mingw\_017.exe}をダブルクリックします。
後はメッセージに従ってインストールを行います。

\item{mingw\_toolsのインストール}

ダウンロードした{\it mingw\_tools\_004.exe}をダブルクリックします。
後はメッセージに従ってインストールを行います。

\item{open-cobolのインストール}

ダウンロードした{\it open-cobol-0.10.1\_011.exe}をダブルクリックします。
後はメッセージに従ってインストールを行います。

\end{itemize}

\subsubsection{COBOL Webインタフェースのインストール}

当資料で紹介しているCOBOL Webインタフェースをインストールします。
COBOL Webインタフェースはソースモジュールによる配布としております。

{\it http://pulse.homeip.net/webif.html}より{\it web.0.2.tar.gz}を{\it c:¥temp}へダウンロードします。

デスクトップにある{\it Msys}アイコンをクリックします。

起動されたMsysウィンドウで、以下のコマンドを実行します。

\$ mv /c/temp/web.0.2.tar.gz .

\$ tar zxvf web.0.2.tar.gz

\$ make libweb.a

\$ cp -p libweb.a /usr/local/lib

以上でインストールは終了です。

/usr/local/lib/libweb.aは実際にはc:¥msys¥1.0¥local¥lib¥libweb.aにコピーされます。






