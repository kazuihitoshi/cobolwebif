
CGIで処理時間を必要とする処理を行う場合、処理が終了するまで再リクエストを遮断する必要があります。

CGIの場合、再リクエストがなされると前回のリクエストの処理はそのままに(処理中であっても)、
再リクエストに対する処理が起動されます。

なかなかレスポンスが無い場合、オペレータはボタンを連打するなどの反応を示す場合があります。
この場合、永久にレスポンスは返らない事になります。

再リクエストを遮断する方法は多種多様であるが、ここではJAVA Scriptを使用し、
Webブラウザ側で再リクエストを遮断する方法を紹介します。

JAVA Scriptで再リクエストを遮断する場合、処理中に別の画面を表示する手法で行います。

処理概要としては、html上にてリクエスト処理中の画面を非表示モードで定義しておき、
リクエスト時に表示する手順で行います。

具体的には以下の様な内容になります。

\vspace{1em}

{\gt 再リクエストの遮断サンプル}

{\footnotesize
\begin{tabular}{|l|}
\hline
\verb+<html LANG="ja" DIR="LTR">+\\
\verb+<meta http-equiv="Content-Type" CONTENT="text/html; charset=X-EUC-JP">+\\
\verb+<script language="JavaScript">+\\
\verb+submit_block_flag = true;+\\
\verb+function submit_block(){+\\
\verb+   document.getElementById('mainform').style.display = 'none';+\\
\verb+   document.getElementById("message").style.visibility="visible";+\\
\verb+   window.defaultStatus = "要求を送信しています。" +\\
\verb+   return true;+\\
\verb+}+\\
\verb+</script>+\\
\verb+<form action="CGIプログラム名" method="POST" onSubmit="submit_block()">+\\
\verb+<span id=mainform>+\\
\verb+ <input type="text" name="DATA1" maxlength="20" size="25" value="@DATA1">  ←ここに画面内容を記述します。+\\
\verb+ <input type="text" name="DATA2" maxlength="20" size="25" value="@DATA2">+\\
\verb+ <input type="text" name="DATA3" maxlength="20" size="25" value="@DATA3">+\\
\verb+ <input type="text" name="DATA4" maxlength="20" size="25" value="@DATA4">+\\
\verb+ <input type="submit">+\\
\verb+ <input type="reset">+\\
\verb+</span>+\\
\verb+<span id="message" style="visibility:hidden;">+\\
\verb+<h2>処理中です。</h2>    ←ここに処理中に表示する内容を記述します。+\\
\verb+</span>+\\
\verb+</form>+\\
\verb+</html>+\\
\hline
\end{tabular}
}

説明しますと、mainformで本来の画面を定義しておき、messageで処理中の画面を定義しておきます。
messageはstyle="visibility:hidden;"にて非表示にしておきます。

リクエストがなされたとき(OnSubmitイベント発生時)にsubmit\_block()関数が実行され、
mainformが非表示となり、messageが表示される。この時点で再リクエストの遮断が成立することとなります。

後はCGI側のリクエストHTMLで、mainformとmessageの表示非表示関係を制御してやれば良いことになります。

もちろんこの方式ですと、JAVA Scriptが実行可能なWebブラウザでなければなりません。また、spanも解釈可能でなければなりませんので、万能というわけではないかもしれませんが、筆者の環境(IE 5.0,moziila 1.0)では正常に動作しています。


