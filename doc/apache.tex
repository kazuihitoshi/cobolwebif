ほとんどのLinuxディストリビューションではApacheのインストールが完了しているので、
当資料ではインストールについては記述しない。

インストール後に必要な設定のみ記述します。

当資料のプログラム環境を整えるためにはApacheに、CGI実行,ユーザディレクトリ,別名定義モジュールを組み込む必要がある。

\vspace{1em}
{\gt 組み込みが必要なApacheモジュール}

\begin{tabular}{|l|l|}
\hline
モジュール名 & モジュールファイル名 \\
\hline
CGI実行モジュール		&mod\_cgi.so	\\
ユーザディレクトリモジュール	&mod\_userdir.so\\
別名定義モジュール		&mod\_alias.so	\\
\hline
\end{tabular}

実際の設定は以下の様になる。(以下の設定例ではapacheのバージョンを1.3としています。)

FileName:/etc/apache/httpd.conf

\begin{tabular}{|l|}
\hline
\verb+LoadModule cgi_module /usr/lib/apache/1.3/mod_cgi.so+\\
\verb+LoadModule userdir_module /usr/lib/apache/1.3/mod_userdir.so+\\
\verb+LoadModule alias_module /usr/lib/apache/1.3/mod_alias.so+\\
\hline
\end{tabular}

または

FileName:/usr/local/etc/apache/httpd.conf

\begin{tabular}{|l|}
\hline
\verb+LoadModule cgi_module /usr/local/lib/apache/1.3/mod_cgi.so+\\
\verb+LoadModule userdir_module /usr/local/lib/apache/1.3/mod_userdir.so+\\
\verb+LoadModule alias_module /usr/local/lib/apache/1.3/mod_alias.so+\\
\hline
\end{tabular}

%{\gt ユーザ登録}
%
%COBOL開発用のユーザを登録する必要はありませんが、
%当資料の便宜上として{\gt cobol}ユーザを登録します。
%
%以下のコマンドは、root(スーパユーザ)にて行ってください。
%
%\begin{tabular}{|l|}
%\hline
%\verb&# useradd -m -d /home/cobol cobol&\\
%\verb&# passwd cobol&\\
%\verb&Enter new UNIX password: &\\
%\verb&Retype new UNIX password: &\\
%\verb&passwd: password updated successfully&\\
%\hline
%\end{tabular}
%
%Enter new UNIX password:にcobolユーザのパスワードを入力してください。
%
%Retype new UNIX password:にはEnter new UNIX passwordにて入力したパスワードをもう一度入力してください。
 

