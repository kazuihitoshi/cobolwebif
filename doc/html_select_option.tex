\begin{itemize}

\item{表記方法}

\begin{tabular}{l}
\verb+<select name="CGI変数名">+\\
\verb+<option value="変数内容" @変数名称:変数内容:SELECTED:>表示コード 表示名称</option>+\\
\verb+<option value="変数内容" @変数名称:変数内容:SELECTED:>表示コード 表示名称</option>+\\
   : : \\
\verb+</select>+\\
\end{tabular}

\item{説明}

プルダウンメニューを定義します。

\item{コーディング例}

{\gt html}

\begin{tabular}{|l|}
\hline
\verb+<HTML LANG="ja" DIR="LTR">+\\
\verb+<META HTTP-EQUIV="Content-Type" CONTENT="text/html; charset=X-EUC-JP">+\\
\verb+<FORM ACTION="test" METHOD="POST">+\\
\verb+血液型<select name="BLOODTYPE">+\\
\verb+<option value="o"  @BLOODTYPE:o:SELECTED:>o  O型 </option>+\\
\verb+<option value="a"  @BLOODTYPE:a:SELECTED:>a  A型 </option>+\\
\verb+<option value="b"  @BLOODTYPE:b:SELECTED:>b  B型 </option>+\\
\verb+<option value="ab" @BLOODTYPE:ab:SELECTED:>ab AB型</option>+\\
\verb+</select>+\\
\verb+<input type="submit" value="実行">+\\
\verb+</form>+\\
\verb+</html>+\\
\hline
\end{tabular}

{\gt cobol} source test.cob

\begin{tabular}{|l|}
\hline
\verb+ working-storage section.+\\
\verb+  01 web-if-rec.+\\
\verb+   02 filler pic x(20) value 'BLOODTYPE'.+\\
\verb+   02 filler pic x value low-value.+\\
\verb+   02 filler pic x value space.+\\
\verb+   02 filler pic x value low-value+\\.
\verb+   02 BLOODTYPE pic x(2) value space.+\\
\verb+   02 filler pic x value low-value.+\\
\verb+   02 filler pic x value low-value.+\\
\verb++\\
\verb+ procedure division.+\\
\verb+ main.+\\
\verb+  call 'WEB_QUERY_STRING'.+\\
\verb+  call 'WEB_POP' using web-if-rec.+\\
\verb+* 処理+\\
\verb+  call 'WEB_PUSH' using web-if-rec.+\\
\verb+  call 'WEB_SHOW'.+\\
\verb+  stop.+\\
\hline
\end{tabular}

\end{itemize}



