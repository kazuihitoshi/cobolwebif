CGIにてWebブラウザから送信されるデータは以下の様な形式です。

送信されたデータが

\begin{tabular}{|l|l|}
\hline
変数名 & 内容\\
\hline
d1     &あいうえお \\
d2     &かきくけこ\\
\hline
\end{tabular}

の内容の場合、送信データは

\begin{tabular}{|l|}
\hline
\verb+d1=%A4%A2%A4%A4%A4%A6%A4%A8%A4%AA&d2=%A4%AB%A4%AD%A4%AF%A4%B1%A4%B3+\\
\hline
\end{tabular}

となります。(ページの日本語コードをeucコードと仮定しています)

送信データをCOBOLプログラム的に上記を扱いやすい形式に編集するには「\&」で分割し、

\begin{tabular}{|l|}
\hline
\verb+d1=%A4%A2%A4%A4%A4%A6%A4%A8%A4%AA+\\
\hline
\verb+d2=%A4%AB%A4%AD%A4%AF%A4%B1%A4%B3+\\
\hline
\end{tabular}

\vspace{1em}

\hspace{-1em}\%付でデコードされた情報をエンコードする必要があります。

\begin{tabular}{|l|}
\hline
\verb+d1=%A4%A2%A4%A4%A4%A6%A4%A8%A4%AA→a4a2a4a4a4a6a4a8a4aa(あいうえお)+\\
\hline
\verb+d2=%A4%AB%A4%AD%A4%AF%A4%B1%A4%B3→a4aba4ada4afa4b1a4b3(かきくけこ)+\\
\hline
\end{tabular}

\vspace{1em}

COBOLで試しにコーディングしてみると

\begin{verbatim}
   :       :
 working-storage section.
 01 query-string-rec.
   02 d1   pic x( .. ).
   02 d2   pic x( .. ).
 01 query-string-data.   ←このエリアに送信データが入ると仮定
   02 filler pic x( .. ). 
 procedure division.
 main  section.
 00.
   unstring query-string-data delimited by "&"
    into d1 d2
   end-unstring   
   d1をエンコード
   d2をエンコード
   :       :
   本来の処理を実行
   :       :
   display 'Content-Type:text/html'      ←結果をHTMLで表示
   display ''
   display '<html>'
      :      :
   display '</html>'
 99.
   :       :
\end{verbatim}

かなり省略してしまいましたが(エンコードの所など)、だいたい上記の様になります。

この方法は全部COBOLソースで記述出来てしまうので、COBOLさえ動けば何とかすることが出来るでしょう。

ですが、ここでもうちょっとひねってみてください。上記だと入力項目毎にエンコード行を書いていかないといけません。
1〜10項目ぐらいであれば上記も良いでしょうが、入力項目が多くなると少々つらくなってきます。
COBOLプログラムの中にHTMLが入り込んでいるのも(別言語が混ざっている感じがして)、少々きになります。

筆者としては、COBOLは階層化されたデータ構造をまとめてえいやーーと処理出来てしまうあのダイナミック(がさつさ?)さを捨てたくありません。
また、HTMLは表示様式であるためプログラムとは別にしておいた方がモジュールの独立性上、良いようにおもわれます。

たとえば、処理をプログラマが作成してHTMLはWebデザイナー専門の人に作ってもらう等、それぞれが独立していれば分業が可能です。

一人の作業時も処理実装時には簡易的なHTMLで実装し、後で見た目(HTML)だけを作成し直すことも可能なのです。
(逆もまたしかりです)

ということで筆者はCOBOLをWebで使用するためのAPIを作成しました。

このAPIを使用すれば、

\begin{itemize}
\item{ エンコード,デコード }
\item{HTMLファイルとCOBOL出力データの合成}
\item{内部変数のページへの保存(hidden表示)}
\end{itemize}

等、実際の処理とは関係ない部分を代わりに処理してもらえます。

