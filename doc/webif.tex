当資料で使用するCOBOL Webインタフェースのインストール方法を紹介します。

COBOL Webインタフェースは{\it http://www.pulse.homeip.net/webif.html}よりダウンロード出来ます。

{\it http://www.pulse.homeip.net/webif.html}より{\it web.0.1.tar.gz}をダウンロードし、以下の手順でコンパイルします。

\begin{tabular}{|l|}
\hline
\verb+# mkdir web.0.1+\\
\verb+# cd web.0.1+\\
\verb+# tar zxvf ../web.0.1.tar.gz+\\
\verb+# make+\\
\verb+# su+\\
\verb+Password: スーパユーザのパスワードを入力+\\
\verb+# make install+\\
\hline
\end{tabular}

以下のファイルがインストールされます。

\begin{tabular}{ll}
/usr/local/lib/libweb.a       &ライブラリ本体\\
\end{tabular}

尚、make install にて以下モジュールもコンパイルされています。

cgi-bin以下にコピーしてもらえれば、動作させてみることが出来ます。

\begin{tabular}{ll}
./addsample        &足し算サンプル\\
sample/addsample.html &足し算サンプルのhtml\\
./calcsample       &四則演算サンプル\\
sample/calcsample.html &四則演算サンプルのhtml\\
\end{tabular}


注意:./configureにてエラーが発生した場合、必要なモジュールを組み込んでください。









