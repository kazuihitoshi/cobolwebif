
COBOLは一昔前の言語であることは否めませんが、現在も事務処理を行うには適した特徴をたくさん持っております。

当資料及びAPIは、この古めかしさも否めないCOBOLをWebベースで使ってしまおうという狙いを持っています。

本来であれば、COBOLを知らない方にとっても有用な資料としたいところですが、今回は多少なりともCOBOLを知っている方に絞って記述します。

COBOLプログラムをWebから使うのは実は非常に簡単です。

その理由は、WebのCGI設計者が既存のプログラムを出来るだけ変更無しに使えるように設計されているためです。

ちょっと具体的な点に触れてみましょう。

WebのCGIはWebブラウザとWebサーバの2人三脚で動作します。

Webブラウザからの送信情報はWebサーバへ送られ、該当するプログラムを起動します。

\par

プログラムは処理をHTMLで標準出力へ出力します。

標準出力へ出力されたHTMLは、Webサーバを通してWebブラウザへ返信され、表示がなされます。

\vspace{1em}

\begin{tabular}{|lllll|}
\hline
クライアント側 & & サーバ側 & & \\
\hline
Webブラウザ & → & Webサーバ & → & プログラム \\
            &    &           &    &  Webブラウザから送信された内容を受信 \\
            &    &           &    &  標準入力と環境変数のどちらかで受信する\\
            &    &           &    &       ↓ \\
            &    &           &    &  プログラム内部で処理\\
            &    &           &    &       ↓ \\
            &    &           &    &  処理結果をHTMLで表示(標準出力へ出力)\\
返信内容が表示される& ← &  ←       &←  &  ←表示と同時に表示内容が返信される \\
\hline
\end{tabular}

\vspace{1em}

どうでしょう。COBOLでどのようにコーディングすればよいかイメージ出来た方もいらっしゃるのではないでしょうか?

\par

そう、基本的にはacceptして処理してdisplayすればよいのです。

おお、そうなんだと思われた方はちょっと待ってください。
Webブラウザからの送信情報のうち日本語等はエンコードされていたり、
COBOLでそのまま使うには少々困難な情報です。
といっても、がっかりしないでください、この資料で紹介するWebAPIを使えば非常に簡単に扱いやすい形で行うことが出来ます。

当資料で使用するCOBOL処理系はOpenCOBOL\footnote{OpenCOBOLは西田氏作成のフリーコンパイラです。}または TinyCobol\footnote{TinyCobolはAndrewCameron氏他作成のフリーコンパイラです}です。
OSはUNIXクローン(筆者の環境はDebianLinux Woody)やMigGwを利用したWindowsでCOBOL処理系とApacheが動作する環境であれば
どんな環境でも動作するとおもいます。

この資料とWeb APIにて、一人でも多くの方にCOBOLでWebプログラミングをお楽しみ頂けることを望んでおります。

それでは本編をお楽しみください。








