{\gt html} addsample.html

{\footnotesize 
\begin{tabular}{|l|}
\hline
\vspace{-0.4em}\verb!<html lang="ja" dir="ltr">!\\
\vspace{-0.4em}\verb!<meta http-equiv="content-type" content="text/html;charset=x-euc-jp">!\\
\vspace{-0.4em}\verb!<form action="addsample.cgi" method="post">!\\
\vspace{-0.4em}\verb!足し算処理のサンプルです。<br><br>!\\
\vspace{-0.4em}\verb!!\\
\vspace{-0.4em}\verb!<input type="text" name="DATA1" maxlength="12" size="15" value="@DATA1">!\\
\vspace{-0.4em}\verb!+!\\
\vspace{-0.4em}\verb!<input type="text" name="DATA2" maxlength="12" size="15" value="@DATA2">!\\
\vspace{-0.4em}\verb!=!\\
\vspace{-0.4em}\verb!@RESULT!\\
\vspace{-0.4em}\verb!<input type="submit" value="実行">!\\
\vspace{-0.4em}\verb!<input type="reset" value="取消">!\\
\vspace{-0.4em}\verb!</form>!\\
\vspace{-0.4em}\verb!</html>!\\
\\
\hline
\end{tabular}
}

\vspace{1em}
{\gt cobol source} addsample.cob

{\footnotesize
\begin{tabular}{|l|}
\hline
\vspace{-0.4em}\verb!       IDENTIFICATION        DIVISION.!\\
\vspace{-0.4em}\verb!       PROGRAM-ID.           addsample.!\\
\vspace{-0.4em}\verb!       AUTHOR.               kazui.!\\
\vspace{-0.4em}\verb!       ENVIRONMENT           DIVISION.!\\
\vspace{-0.4em}\verb!       CONFIGURATION         SECTION.!\\
\vspace{-0.4em}\verb!       INPUT-OUTPUT          SECTION.!\\
\vspace{-0.4em}\verb!       DATA                  DIVISION.!\\
\vspace{-0.4em}\verb!       FILE                  SECTION.!\\
\vspace{-0.4em}\verb!      *!\\
\vspace{-0.4em}\verb!       WORKING-STORAGE SECTION.!\\
\vspace{-0.4em}\verb!       01 webif-rec.!\\
\vspace{-0.4em}\verb!         02 filler pic x(20) value 'DATA1'.!\\
\vspace{-0.4em}\verb!         02 filler pic x value low-value.!\\
\vspace{-0.4em}\verb!         02 filler pic x value space.!\\
\vspace{-0.4em}\verb!         02 filler pic x value low-value.!\\
\vspace{-0.4em}\verb!         02 data1 pic x(12) value space.!\\
\vspace{-0.4em}\verb!         02 filler pic x value low-value.!\\
\vspace{-0.4em}\verb!         02 filler pic x(20) value 'DATA2'.!\\
\vspace{-0.4em}\verb!         02 filler pic x value low-value.!\\
\vspace{-0.4em}\verb!         02 filler pic x value space.!\\
\vspace{-0.4em}\verb!         02 filler pic x value low-value.!\\
\vspace{-0.4em}\verb!         02 data2 pic x(12) value space.!\\
\vspace{-0.4em}\verb!         02 filler pic x value low-value.!\\
\vspace{-0.4em}\verb!         02 filler pic x(20) value 'RESULT'.!\\
\vspace{-0.4em}\verb!         02 filler pic x value low-value.!\\
\vspace{-0.4em}\verb!         02 filler pic x value space.!\\
\vspace{-0.4em}\verb!         02 filler pic x value low-value.!\\
\vspace{-0.4em}\verb!         02 result pic x(12) value space.!\\
\vspace{-0.4em}\verb!         02 filler pic x value low-value.!\\
\vspace{-0.4em}\verb!         02 filler pic x value low-value.!\\
\vspace{-0.4em}\verb!       01  webif-in.!\\
\vspace{-0.4em}\verb!         02 data1  pic 9(12).!\\
\vspace{-0.4em}\verb!         02 data2  pic 9(12).!\\
\vspace{-0.4em}\verb!         02 result pic 9(12).!\\
\vspace{-0.4em}\verb!       01  webif-ot.!\\
\vspace{-0.4em}\verb!         02 data1  pic zzzzzzzzzzz9.!\\
\vspace{-0.4em}\verb!         02 data2  pic zzzzzzzzzzz9.!\\
\vspace{-0.4em}\verb!         02 result pic zzzzzzzzzzz9.!\\
\vspace{-0.4em}\verb!      *------------------------------------------------!\\
\vspace{-0.4em}\verb!       PROCEDURE             DIVISION.!\\
\vspace{-0.4em}\verb!       MAIN-PROC             SECTION.!\\
\vspace{-0.4em}\verb!       MAIN-PROC-1.!\\
\vspace{-0.4em}\verb!      *!\\
\vspace{-0.4em}\verb!          call 'WEB_GET_QUERY_STRING'.!\\
\vspace{-0.4em}\verb!          call 'WEB_POP' using webif-rec.!\\
\vspace{-0.4em}\verb!          move corr webif-rec to webif-in.!\\
\vspace{-0.4em}\verb!          compute result of webif-in = data1 of webif-in +!\\
\vspace{-0.4em}\verb!                                       data2 of webif-in .!\\
\vspace{-0.4em}\verb!          move corr webif-in to webif-ot.!\\
\vspace{-0.4em}\verb!          move corr webif-ot to webif-rec.!\\
\vspace{-0.4em}\verb!          call 'WEB_PUSH' using webif-rec.!\\
\vspace{-0.4em}\verb!          call 'WEB_SHOW'.!\\
\vspace{-0.4em}\verb!      *!\\
\vspace{-0.4em}\verb!          STOP RUN.!\\
\vspace{-0.4em}\verb!!\\
\\
\hline
\end{tabular}
}

{\gt メニューHTML(起動用のHTML定義)}

\begin{tabular}{|l|}
\hline
\verb+<html lang="ja" dir="ltr">+\\
\verb+<meta http-equiv="content-type" content="text/html;charset=x-euc-jp">+\\
\verb+<a href="/cgi-bin/addsample?screenname=addsample.html">足し算サンプル</a>+\\
\verb+</html>+\\
\hline
\end{tabular}